%
% Copyright (C) 2001, Sun Microsystems, Inc
%
% See the file "LICENSE.TERMS" for information on usage and
% redistribution of this file, and for a DISCLAIMER OF ALL
% WARRANTIES.
%

\documentclass[11pt,letter]{book}
\usepackage{times}

% attempting to change margins, but top lines may get messed up
% Seems to work OK now... needed to be before call to fancyheadings
\textwidth 6.0in
\oddsidemargin  0.125in 
\evensidemargin  -0.1875in

% Make paragraph indentation zero and add extra space between paragraphs
\setlength{\parindent}{0cm}
\setlength{\parskip}{0.5em}

% header and footer layout
\usepackage{inputs/fancyheadings}
\usepackage{ifthen}
\pagestyle{fancy}
\renewcommand{\chaptermark}[1]{\markboth{#1}{#1}}
\renewcommand{\sectionmark}[1]{\markright{\thesection\ #1}}
\lhead[\fancyplain{}{\bf\thepage}]{\fancyplain{}{\bf\rightmark}}
\rhead[\fancyplain{}{\bf\leftmark}]{\fancyplain{}{\bf\thepage}}
\cfoot{}

% verbatim input of files and postscript files
\usepackage{verbatim}	% needed for smallverbatim and fnverbatim
\newenvironment{smallverbatim}{\small\verbatim}{\endverbatim}
\newenvironment{fnverbatim}{\footnotesize\verbatim}{\endverbatim}

\usepackage{graphics}
\usepackage{url}
\usepackage{epsfig}

% allow figures of 90% the size of the page, and 10% text on the page
\renewcommand{\floatpagefraction}{0.9}
\renewcommand{\topfraction}{0.9}
\renewcommand{\bottomfraction}{0.1}
\renewcommand{\textfraction}{0.1}

% try to minimize the number of orphans and widows
\pretolerance=500
\tolerance=10000
\brokenpenalty=10000
\widowpenalty=10000
\clubpenalty=10000

% figure commands
\newcommand{\centerfigbegin}{\begin{figure}[htp] \begin{center}}
\newcommand{\centerfigend}[2]{{\small \caption{\label{#1} {#2}}} \end{center} \end{figure} }

\newcommand{\psfigbegin}[2]{\begin{figure}[htp] \centerline{\psfig{figure={#1},height={#2}}} }
\newcommand{\psfigend}[2]{{\small \caption{\label{#1} {#2}}} \end{figure} }

% new latex environments
\newtheorem{definition}{Definition}

% useful shortcuts 
\newcommand{\walk}{Walkabout}
\newcommand{\uqbt}{{\bf UQBT}}

\makeindex


\begin{document}
\title{\Large \bf Walkabout \\ A Framework for Experimental Dynamic 
	Binary Translation}
\author{
{\bf Investigators} \\
	Cristina Cifuentes \\
	Brian Lewis \\ \\
{\bf Interns} \\
	May - Aug 2001: David Ung \\
	May - Aug 2001: Bernard Wong \\
	Jan - Jun 2001: Nathan Keynes 
}
\date{
	January 2002 \\ 
    \copyright 2001-2002, Sun Microsystems, Inc }
\maketitle


\section*{Abstract}

Dynamic compilation techniques have found a renaissance in recent  
years due to their applicability to performance improvements 
of running Java-based code. 
Techniques originally developed for object-oriented language virtual 
machines are now commonly used in Java virtual machines and Java 
just-in-time compilers. 
Application of such techniques to the process of binary translation 
has also been done in recent years, mainly in the form of 
binary optimizers rather than binary translators. 

The \walk\ project proposes dynamic binary translation techniques 
based on properties of retargetability, ease of experimentation, 
separation of machine-dependent from machine-independent concerns, 
and good debugging support.  
\walk\ is a framework for experimenting with dynamic binary
translation ideas, as well as ideas in related areas, such as 
interpreters, instrumentation tools, optimization and more. 


% If you want all the chapters to be included, set the boolean "all" to be 
% true, else, list the chapter you are interested in in the "else" brackets.

\newboolean{all}
\setboolean{all}{true}

\ifthenelse{\boolean{all}}
{
\tableofcontents
\listoffigures

	\chapter*{Preface}

The \walk\ project was an initial investigation conducted
in 2001 at Sun Labs in relation to dynamic binary translation. 
The project ran for 9 months and was led by Cristina Cifuentes, 
with input from 3 different interns; Nathan Keynes, David Ung
and Bernard Wong, and one other researcher, Brian Lewis. 

\walk\ was built based on ideas from the UQBT project (see 
\url{http://www.itee.uq.edu.au/csm/uqbt.html}, and as such 
aimed at retargetable experimentation through specification of 
features of machines.  In fact, the New Jersey Machine Code 
SLED specifications and the UQBT SSL specifications were reused
to automatically generate disassemblers and interpreters for the 
SPARC and x86 architectures.   

We hope that others will make use of the framework; many types
of experimentation can be done with the current design and 
its partial implementation. 


{\small
\begin{flushright}
Cristina Cifuentes \\
Mountain View, California \\
14 Jan 2002
\end{flushright}
}




	\chapter{Introduction}
\label{ch-intro}

{\small
\begin{flushright}
Design: Cristina [2001]; Documentation: Cristina, Brian [Jan 2002]
\end{flushright}
}


Binary translation, the process of translating binary executables\footnote{
In this document, the terms \emph{binary executable}, \texttt{executable},
and \texttt{binary} are used as synonyms to refer to the binary image
file generated by a compiler or assembler to run on a particular
computer.
} makes it possible to run code compiled for source
platform M$_s$ on destination platform M$_d$.  Unlike an interpreter or
emulator, a binary translator makes it possible to approach the speed
of native code on machine M$_d$.  Translated code may run more slowly than
native code because low-level properties of machine M$_s$ must often be
modeled on machine M$_d$.  For example, the Digital Freeport Express
translator~\cite{Dec95} simulates the byte order of SPARC architecture, 
and the FX!32 translator~\cite{Thom96,Hook97} simulates the calling sequence 
of the source x86 machine, even though neither of these is native to the 
target Alpha architecture.

The \walk\ framework is a retargetable, dynamic binary translation framework 
for experimentation with dynamic translations of binary code.  
The framework grew out of the UQBT framework~\cite{Cifu00e,Cifu01e,Cifu01f}, 
by taking what we had learned in the areas of retargetability of binary code 
and separation of machine-dependent from machine-independent concerns, and 
applying such techniques to the new dynamic framework.  Clearly, the choice of
transformations on the code would need to be different due to the differences
between dynamic and static translations.  


\section{Goals and Objectives}

Binary translation requires machine-level analyses to transform 
source binary code onto target binary code, either by emulating 
features of the source machine or by identifying such features and 
transforming them into equivalent target machine features. 
In the \walk\ system we plan to make use of both types of transformations, 
determining when it is safe to make use of native target features. 

One question that is hard to answer before experimenting in a system 
is that of the choice of intermediate representation. 
In the UQBT system we made use of RTLs and HRTLs; the former being a 
register transfer language that made explicit every transfer of control, 
and the latter being a high-level register transfer language that 
resembled simple imperative languages, where control transfers are 
made explicit.  It is unusual for a binary translation system to 
make use of two different representations for instructions.  

Other binary translation systems have made use of the assembly language 
as the intermediate representation, mainly due to the fact that such 
systems were generating code for the same machine (i.e. they were optimizers
of binary code rather than binary translators per se).  Such 
systems include Dynamo~\cite{Bala00}, Wiggins/Redstone~\cite{Reev00}, 
and Mojo~\cite{Chen00}. 

For \walk, we initially use assembly language and we plan to use
RTL as the next step, though we would like to experiment with its 
suitability and ease of translation into a target representation, 
after all, RTLs are still machine-dependent.  

The goals of the project are: 
\begin{itemize}
\item to derive components of binary translators from machine descriptions,
\item to understand how to instrument interpreters in a retargetable way, 
\item to determine whether an RTL representation is best suited for machine 
	translation, and how to best map M$_s$-RTLs to M$_t$-RTLs, 
\item to understand how debugging support needs to be integrated in 
	a dynamic binary translation system, and 
\item to develop a framework for quick experimentation with ideas in the 
	dynamic binary-manipulation area . 
\end{itemize}

We limit binary translation to user-level code and to multiplatform 
operating systems such as Solaris and Linux.  


\section{Architecture}

The architecture of the \walk\ framework borrows from the architecture 
of most existing dynamic compilation systems such as those for the 
object-oriented languages Smalltalk~\cite{Gold83,Deut84},  
SELF~\cite{Unga87,Holz94} and Java~\cite{Holz97,Grie00,Pale01}.
The idea is simple.  Based on the premise that most programs spend 
90\% of the time in 10\% of the code, the dynamic compilation 
system should only consider compiling that 10\% of the code and 
interpret the rest of the code base, as it is not executed too often. 

\psfigbegin{figures/walkaboutArchitecture.eps}{8cm}
\psfigend{fig-walkabout}{The Architecture of the Walkabout Framework}

Figure~\ref{fig-walkabout} illustrates the architecture of the system. 
The source binary program is initially interpreted until a hot path 
is found.  Code is generated for that hot path and placed into a 
(translated) instruction cache (called fragment cache or F\$ in our 
notation).  Once the generated code is executed, control transfers to 
the interpreter to interpret more code, and the process repeats for pieces 
of code that have not been interpreted or translated as yet.  
For pieces of code that have been translated, the translated version in 
the instruction cache is executed instead. Further, if a particular 
piece of translated code is executed too often, the code can be 
reoptimized and new, efficient code can be generated. 

\psfigbegin{figures/walkabout2001.eps}{8cm}
\psfigend{fig-walkabout2001}{The 2001 Walkabout Framework}

The 2001 \walk\ implementation does not implement the complete 
framework.  This document and the present open source release are
the results of a 9-month experiment conducted with interns, 
hence, only parts of the system are in place.  
Figure~\ref{fig-walkabout2001} illustrates the 2001 \walk\ implementation. 
As can be seen, code generation was for the same family of machines,
the SPARC architecture in this case, where SPARC V9 code was generated 
for SPARC V8 source binaries.  


\subsection*{Applications of the \walk\ Framework}

The \walk\ framework can be used to build dynamic translators
having a wide range of uses.
For example, it supports the construction of
analysis and instrumentation tools
that insert code during translation
in order to understand the behavior of running programs.
These tools can do basic block counting and profiling.
They can also record dynamic memory accesses,
branches taken or not, and instruction traces.
The data they collect can be used to drive related tools
such as pipeline and memory system simulators.
Systems using dynamic translation for instrumentation include
ATOM~\cite{Sriv94b} and Vulcan~\cite{Sriv01}.

\walk\ can also be used to build optimizers:
dynamic translators that improve the performance of programs.
Several examples of program optimizers were given above.
Also, Schnarr and Larus~\cite{Schn96} describe how
rescheduling legacy code for newer processors with different pipelines
can significantly improve performance.
Other applications of \walk\ include machine emulators and program checkers.
Machine emulators give executing programs
the illusion that they are running on a different machine.
They can be used to run legacy programs on newer hardware
and to simulate new machines on existing hardware.
An example of the latter is the Daisy~\cite{Ebci96} system.
Program checkers execute programs while continuously checking
that they operate correctly or safely:
for example, that they reference only allowed memory locations,
or execute only allowed system calls.
As an example of the latter,
the STRATA~\cite{Scot01} dynamic translation system has been used to
enforce a number of different software security policies.


\subsection{The Interpreter Generator}

We always thought that the UQBT machine descriptions for syntax and 
semantics of machine instructions were complete enough to support 
the generation of interpreters for user-level code.
The user-level code restriction is imposed by the SSL descriptions, 
which only describe user-level instruction semantics.  This decision 
was inline with the goals of the UQBT project. 

We took the syntactic (SLED) and semantic (SSL) descriptions 
for the SPARC and x86 architectures and experimented with the 
automatic generation of interpreters for these two machines.  
Figure~\ref{fig-genemu} illustrates the process. 

\psfigbegin{figures/genemu.eps}{4cm}
\psfigend{fig-genemu}{The Interpreter Generator Genemu}

The interpreter generator, \texttt{genemu}, parses SLED and SSL 
specifications for a machine, knows how to decode ELF binary 
files, and generates an interpreter for that machine in the 
C or Java language.  

As reported in Chapter~\ref{ch-genemu}, the C-based interpreters 
were tested against the SPEC95 integer benchmarks, whereas the
Java-based interpreters were tested against smaller benchmarks 
as they took so long to run. 

More explanation about this subsystem is given in Chapter~\ref{ch-genemu}.  
Note that the documentation makes use of the term ``emulator'' 
to refer to the ``interpreter''.


\subsection{The Instrumented Interpreter Generator}

We were interested in experimenting with different ways in which 
we could determine hot paths within an interpreter, hence we 
designed an instrumentation language, INSTR, which was 
used in conjunction with the emulator generator in order to 
generate interpreters that instrumented the code in the way 
specified in our INSTR spec.  In this way, we could 
quickly specify different ways of instrumenting code and automatically
generate interpreters instrumenting in the scheme of choice. 
Figure~\ref{fig-genemu-instr} illustrates the process.  

\psfigbegin{figures/genemu-instr.eps}{4cm}
\psfigend{fig-genemu-instr}{The Instrumented Interpreter Generator}

The instrumented interpreter generator, \texttt{genemu\_i}, is an 
extension of \texttt{genemu}, which parses SLED and SSL machine descriptions, 
as well as the INSTR instrumentation description, and generates an 
interpreter for that machine which would instrument instructions in the 
way specified in the INSTR spec. 
The instrumented interpreter was generated in the C language. 

More explanation about this subsystem is given in Chapter~\ref{ch-genemu-instr}.


\subsection{The PathFinder}

The 2001 \walk\ implementation is what is referred to as the 
\texttt{pathfinder}.  The PathFinder implements Figure~\ref{fig-pathfinder}, 
which interprets SPARC V8 (and a few V9?) instructions, uses one 
of four different instrumentation schemes to determine hot paths, 
and generates SPARC V9 code for those hot paths into a fragment cache. 

\psfigbegin{figures/pathfinder.eps}{8cm}
\psfigend{fig-pathfinder}{PathFinder: The Implementation of the 2001 
		Walkabout Framework}

The PathFinder was tested against some SPEC95 and SPEC2000 benchmarks.  
More explanation about this subsystem is given in Chapter~\ref{ch-pathfinder}. 


\subsection{Debugging Support}

One of the goals of the \walk\ project was to provide for better 
debugging support than its UQBT counterpart.  A debugger was 
built to integrate with the other components of the \walk\ system, 
relying on the automatic generation of the disassembler and the 
interpreter. 

The \walk\ debugger is a Java language GUI tool that provides several 
windows to display the assembly instructions of the program, as well 
as its state (i.e. register contents).  Users can set breakpoints and 
run the program to a given state.   

More information about debugging support is given in Chapter~\ref{ch-debugger}.


\section{Status} 

The 2001 \walk\ implementation consisted of 16 man-months over 
a period of 9 months.  The project was led by Cristina Cifuentes, 
and several interns worked on the implementation of different 
components; namely, Nathan Keynes worked on the emulator generator, 
Bernard Wong worked on the debugger and disassemblers, and David 
Ung worked on the hot path instrumentation and code generator.  
Brian Lewis investigated its debugging and testing support, framework
applications, and helped design portions of the emulator generator.

%%
%% Cristina: 1 day/wk over 9 months => 1.8 man-month
%% Nathan: 6 months
%% Bernard: 3 months
%% David: 4 months
%% Brian: 1.5 man-month
%%  


	% general aims and framework 

	%
% 12 Jul 01 - Nathan: Updated to include all the recent changes
%    May 01 - Bernard: Emulator performance study
%

\chapter{Emulator Generator}
\label{ch-genemu}

{\small
\begin{flushright}
Design: Nathan, Cristina; Implementation: Nathan; Documentation: Nathan [Apr 01]
\end{flushright}
}

This chapter documents the inner-workings of the emulator generator 
for the \walk\ framework.  It describes design issues and concentrates 
on interfaces and usage of the emulator generator tool. 
This chapter documents the initial version of the emulator generator 
rather than the release version (i.e. it has not been fully updated) 
[Cristina, Jan 2002]. 


\section{Design}
The main design goals of building an emulator for \walk, were retargetability,
reuse of existing specifications, and efficiency. This has been achieved by
making use of an emulator generator, which accepts the SLED and SSL 
specifications, and outputs source code for an emulator core, in either the
C or Java language. With the addition of some simple scaffolding, the result 
is an easily retargetable emulation system.  

Another goal for the emulator generator was to allow users to create 
emulators without needing to write them in assembler language
or even know the assembler of the machine where the emulator is to 
be run. 


\section{Using the generator}
This section explains the way an emulator can be generated, 
what its interface is and what to know in order to make changes
to the code generation module. 


\subsection{Invocation}
The emulator generator is called \texttt{genemu} and it is invoked
in the following way: 

\begin{verbatim}
genemu [options] <SLED spec file> <SSL spec file> [outputfile]
Recognized options:
   -c  output C code [default]
   -d  disassembler only (do not generate emulator core)
   -j  output Java code
\end{verbatim}


\subsection{Interface}
 
The generated code exports one fundamental function, 
\texttt{void executeOneInstruction()}, which completely executes the 
instruction pointed to by the current PC. 
The emulator expects a pointer \texttt{mem} to have been
set up to refer to the start of the emulated program's address space.

The CPU registers are exported as \texttt{regs}, which is a 
\texttt{RegisterFile} structure (defined in the generated header file). 
Note that right now it's not always simple to know a-priori how to 
access a given register from outside the core, although \texttt{\%pc} 
should always be \texttt{regs.r\_pc}. 

If the SSL specification makes use of traps, the calling code must also define
a callback, of the form

\begin{verbatim}
   void doTrap( bool cond, int trapNumber );
\end{verbatim}
where \texttt{cond} indicates whether the trap should execute, and 
\texttt{trapNumber} gives the trap id.

A minimal use might look something like
\begin{verbatim}
char *mem;

int main()
{
    mem = loadBinary();
    regs.r_pc = getStartAddress();
    run(); /* Supplied by generated emulator */
}
\end{verbatim}


\subsection{Making the specifications work for you}
In order to do what it's supposed to do, the emulator generator relies on
certain correlations between the SLED and SSL files that it's using. This
section documents those requirements.

The primary requirement is that in general all names must be the same (modulo
case). This means instruction names are the same, and register names are the 
same. Additionally operand names which appear in the SSL file must be defined
as fields in the SLED file, but need not necessarily appear in the actual 
instruction constructor. Also, an operand which is used as a register index 
(i.e. appears inside an r[] expression), must be defined in the SLED as a 
field with register names.

Limitations: Assigning to an operand is not currently supported in Java.


\section{Inside the generator - Maintainer's notes}

\subsection{Roadmap}
The directory structure for the emulator generator code is as 
follows.  

\begin{tabbing}
tools/\=codegen\_java.ccaacc\=\kill
include/\\
  \>codegen.h\>Abstract class declarations for CodeGenApp and CodeGenLanguage\\
  \>codegenemu.h\>Main header for emulator generation\\
  \>codegen\_c.h\> Header file for C language support\\
  \>codegen\_java.h\>Ditto for Java (inherits from C)\\
  \>sledtree.h\>Declaration of the SLED AST classes\\
\\
tools/\\
  \>codegen.cc\>Generic methods for CodeGenApp and CodeGenLanguage\\
  \>codegen\_c.cc\>Methods for C language support\\
  \>codegen\_java.cc\>Ditto for Java\\
\\
  \>gendasm.cc\>  Methods for generating disassembly functions\\
  \>gendecode.cc\>Methods for generating instruction decoders (via NJMC)\\
  \>genemu.cc\>   Main function, and initialization\\
  \>genregs.cc\>  Register structure computations\\
  \>genss.cc\>    Semantic String / instruction handling\\
\\
  \>sledscanner.l\>Lexical specification for SLED files (lex)\\
  \>sledparser.y\>Syntax specification and ast construction for SLED files  (yacc)\\
  \>sledtree.cc\>Methods for SLED AST, and some tree construction support\\
  \>sledtest.cc\>Main function for testing SLED routines\\
  \>match.cc\>Simple code to generate NJMC-style matching statements\\
\\
emu/\\
	\>emumain.cc 		\>The emulator's main \\
	\>sparcmain.java	\>The emulator's main for Java-based version \\
\\
	\>personality.h		\>Base class for the OS Personality \\
	\>personality.cc	\>Implementation of the base Personality class \\
	\>linux.cc			\>Linux personality implementation \\
	\>solaris.cc		\>Solaris personality implementation \\
\\
	\>sparcemu.h		\>Emulator interface file for the SPARC architecture \\
	\>sparcemu.m		\>Emulator implementation file for the SPARC  \\
	\>sparcemu.cc		\>Generated file from sparcemu.m \\
\\
	\>instrsparcstub\_c.cc	\>C language stub methods for the SPARC architecture \\
	\>sparcstub\_c.cc	\>C language SPARC stub methods \\
	\>sparcstub\_java.cc \>Java language SPARC stub methods \\
	\>x86stub\_c.cc		\>C language x86 stub methods \\
\\
	\>sysv.cc			\>SysV loader and process initializer \\
\\
tools/runtime/\\
  \>emuskel.c\>Emulator implementation file skeleton for C \\
  \>emuskel.h\>Emulator interface file skeleton for C\\
  \>emuskel.java\>Emulator implementation file skeleton for Java\\
\end{tabbing}


\subsection{Generator code structure}

Conceptually, the generator is separated out into the following pieces:
\begin{itemize}
  \item Language neutral, abstract application base class (\texttt{CodeGenApp})
  \item Main emulator implementation (\texttt{CodeGenEmu})
  \item Application neutral language support classes (\texttt{CodeGenC}, 
		\texttt{CodeGenJava})
  \item Skeleton files (\texttt{emuskel.c}, \texttt{emuskel.h}, 
		\texttt{emuskel.java}, etc)
\end{itemize}

After parsing it's input, the generator reads in the specification files
given to it, and computes the register structures. Then it reads and processes
each skeleton file in turn, writing to the given output file and substituting 
in the actual code as it goes. 

All actual code is generated by calls through \texttt{lang}, which is an
instantiation of \texttt{CodeGenLanguage}\footnote{At least, all code
\emph{should} be generated in this way. Currently there are still quite a
few places where operators are inserted directly, which would need to be
fixed to support less C-like languages.} The skeleton logic is all
generic, and so is implemented in \texttt{CodeGenApp}.


\subsection{Skeleton files}

The skeleton files are processed by reading them line by line,
substituting for any variables found, and writing out again. Variables are
implemented similarly to the Unix shell, and can be specified as either
\$VARNAME or \$\{VARNAME\}.  Variable names may only contain alphanumeric
or underscore characters.

There is also a simple built-in conditional generation mechanism - 
\begin{verbatim}
@SECTION
 ...
@SECTION
\end{verbatim}
will be generated if and only if the SECTION section is active, whereas
\begin{verbatim}
@!SECTION
 ...
@!SECTION
\end{verbatim}
will be generated if and only if the SECTION section is inactive. In either 
case the conditional directives themselves will not be copied to the output.


\subsection{Generated code structure}

The emulator generator normally creates 2 files - interface (ie .h) and 
implementation (.c). (In the case of the Java language, it obviously only 
generates a single output file).  
The interface file contains some basic typedefs and 
function prototypes, along with the declaration of the main register 
structure (\texttt{RegisterFile}).

The guts are of course in the implementation file - This is roughly divided
into prologue (general macros), disassembler, parameter decoding (mapping 
register parameters to registers, and breaking up complex operands), 
instruction routines (one execute routine per instruction), the main
execute() function (essentially instruction decode and dispatch), and finally
the exported executeOneInstruction() routine, which handles the main
fetch-execute cycle (1 cycle's worth).

The disassembler and decoder both depend heavily on the New-Jersey 
Machine Code toolkit (NJMCTK) to generate the real decoders -- the 
generator itself just produces (long) match statements for these parts, 
in the form of .m (matching) files which NJMCTK translates into .cc files.


\section{Stand-alone emulation}

In order to test the emulator properly (and transitively the
specifications), it's useful to be able to run it in isolation from the
rest of \walk, ie to completely emulate a binary application. The emulator
source generated from this tool is obviously not capable of doing this by
itself - it needs support to load binary files, handle operating system
calls, etc.

Included in the \texttt{emu/} directory is a small set of files to provide the
needed infrastructure. Currently it contains support for Solaris and Linux
platforms (at least partially, more work is needed for completeness), and
stubs for SPARC and x86. In order to support a new CPU core with these
platforms, it is only necessary to generate an emulator with the toolkit,
and write a small stub file. The task of the latter is to supply routines
to set the stack pointer, setup the program counter, and most importantly
handle parameter passing to and from system calls. (We currently make the
unsupported assumption that there is a standard for this on each CPU
architecture - when adding more platforms these stub files will
undoubtedly need to handle multiple conventions).

In the case of a Java-based CPU core, the stub file is also responsible for
thunking certain calls through to the Java runtime environment.


\subsection{Personality}

The Personality class (and subclasses thereof) is responsible for the
loading of binary files, initial stack setup, and syscall handling---in
other words for imitating the normal behaviour of the kernel on a real
system. Personality in itself only supplies a few utility and factory
methods; each subclass is responsible for implementing two key methods:

\begin{itemize}
   \item bool execve( const char *filename, const char **argv, 
		const char **envp )
   \item int handleSyscall( int callno, int *parms )
\end{itemize}

\texttt{execve} behaves exactly as the POSIX standard \texttt{execve} function,
except that the caller retains control, and it does not actually start
running the process (which can be done by executing run() as described
later). \texttt{handleSyscall} is called from the relevant stub file whenever 
the CPU core encounters the architectural equivalent of a SYSCALL instruction,
with the number of the syscall, and an array of up to 6 parameters. The
\texttt{handleSyscall} function should return the result of the call in the 
first parameter, possibly setting the carry flag via 
\texttt{setReg\_CF()}.\footnote{On architectures which expect the parameters 
to be returned unchanged, the stub file is responsible for making a copy of 
them.}

Note that this is indeed somewhat biased towards Unix and Unix-like
systems, however at least currently that includes all systems of interest.
It also seems likely that most other systems could be mapped to make use
of this interface.

In addition to being a base class for OS-specific behaviour, Personality 
supplies some basic functions for dealing with the process memory image - in 
particular all memory accesses from subclasses should be routed through the 
\texttt{putUser*} / \texttt{getUser*} functions, as they ensure correct 
byte ordering.


\subsection{SysVPersonality}

As a convenience, due to the large amount of overlap between modern
Unix-like systems, the class SysVPersonality (\texttt{sysv.cc}) was introduced 
to contain the common parts. This is primarily an implementation of 
\texttt{execve}, which is a largely standard (and rather non-trivial) process 
on systems supporting the ELF file format. Note that the BinaryFile API used
in UQBT is not used here, as rather lower-level information is
needed by the loader. SysVPersonality also creates one new abstract
method
\begin{verbatim}
  int handleAuxv( AUXV_T *auxv )
\end{verbatim}
which permits subclasses to add additional items to the process image's 
auxiliary vector (is passed pointer to first free vector, returns number of 
items added ). 

Note that there are some small machine dependencies which have to do with
the exact stack layout. Adding a new architecture to the emulator may
require adding an entry to the switch here (unfortunately this seems
unavoidable).


\subsection{Stubs}

As previously mentioned, a stub file needs to be written for each CPU 
architecture (and for each language, for that matter). A list of these 
functions is at the top of \texttt{personality.h}, but a quick description 
may be useful

\begin{itemize}
  \item{\texttt{setReg\_pc(int)}} Set the program counter to the given value, 
	so that execution resumes from that point.
  \item{\texttt{setReg\_sp(int)}} Set the stack pointer to the given value
  \item{\texttt{setReg\_CF(int)}} Set the carry flag to true/false if the 
	value is non-zero/zero
  \item{\texttt{initCore()}}     Initialize the processor core (normally a 
	no-op for C cores)
  \item{\texttt{setMem(char *)}} Set the memory base for the emulator core
  \item{\texttt{run()}}          Begin execution - run until told to stop
  \item{\texttt{stop(int)}}      Terminate execute with the given exit value
  \item{\texttt{getArchitecture()}} Return an ID code corresponding to the 
	CPU architecture being emulated
  \item{\texttt{getDefaultPersona()}} Return a personality ID representing 
	the ``default'' platform for a given architecture (ie Solaris for SPARC, 
	Linux for x86)
  \item{\texttt{dumpMainRegisters(FILE *)}} Dump the main CPU registers to the 
	given stream.
\end{itemize}


\section{Performance Analysis}
{\small
\begin{flushright}
Implementation: Bernard; Documentation: Bernard [May 2001]
\end{flushright}
}

This section documents the performance analysis that was done on the 
emulators generated by the emulator generator, with emphasis on the
SPARC emulator.  Descriptions of performance analysis tools and 
results using different tools and techniques are given.  
Note that these experiments were run in May 2001, prior to completion 
of the emulator generator's final form. 


\subsection{Performance Analysis}

From previous performance evaluation work that had been done by Nathan Keynes, 
the approximate performance of the emulator is known 
(see Figure~\ref{fig-nathanBenchC}). 
From this work, we know that the current emulator is approximately 77 times 
slower than a natively executed program.

\centerfigbegin
{
\begin{tabular}{|l|r|r|} \hline
Emulated Program & Slow down from native\\ \hline
\emph{099.go} &66.85x  \\ 
\emph{124.m88ksim} &96.51x\\
\emph{129.compress95} &77.32x\\
\emph{130.li} & 70.1x\\
\emph{132.ijpeg} & 112.49x\\
\emph{134.perl}  & 63.67x \\
\emph{147.vortex} &68.82x\\ \hline
\emph{Mean} & 77.79x \\
\hline
\end{tabular}
}
\centerfigend{fig-nathanBenchC}{Previous performance evaluation of the C++ 
	version of the emulator taken from Nathan Keynes' presentation slides}


\subsubsection{Profiler Breakdown}

To get a better understanding of where the extra time is spent, profiling
of the emulator need to be done. First an appropriate profiler need to be
chosen for the task.

Three different profilers were evaluated - \verb!gprof!, \verb!quantify! and
\verb!Shade!. A brief summary of each tool can be found in 
Figure~\ref{fig-profilerComp}.

\centerfigbegin
{
\begin{tabular}{|l|l|l|} \hline
Gprof				
&Quantify 	
&Shade  \\ \hline

&&\\
\emph{Pros} 		
&\emph{Pros} 	
&\emph{Pros} \\

$\bullet$ Easy to setup and use		& $\bullet$ Gives a nice graphical 	& $\bullet$  Poweful tool for creating  \\
$\bullet$ Gives information of time     & call graph   				& custom profilers \\
	  spent in each function	& $\bullet$ Very detail breakdown of 	& $\bullet$ Can be used to analyse \\
$\bullet$ Shows function trace of the   &function usage				& specific instructions or \\ 
	  application			& 					& instruction sets \\
	  
&&\\
\emph{Cons}				& \emph{Cons}			& \emph{Cons} \\

$\bullet$ Groups time spent in each  	& $\bullet$ Proprietary tool where & $\bullet$ Very large amount of work\\  
function with time spent in	 	& the data gathered can only 	& required to create custom \\
function's children			& easily be viewed within	& profiler\\
$\bullet$ Text based function 		& the program			&\\
trace very hard to follow		&				&\\
&&\\

\hline \end{tabular} } \centerfigend{fig-profilerComp}{Comparison and
evaluation of different profilers for use with the C++ based emulator}

From the limited profiling requirements of the project, a tool such as 
\verb!Shade! is far more complex and time consuming than necessary. 
Most of the important information from profiling are given by much easier 
to use tools such as \verb!gprof! and \verb!quantify!.

Although \verb!gprof! gives enough information to meet most of the profiling 
needs for the emulator, \verb!quantify! can give the same information as
\verb!gprof! but in greater detail and in a graphical environment. The 
proprietary nature of the \verb!quantify! tool is not a large concern as 
the data set gathered does not need to be further analysed by other tools. 
Therefore, \verb!quantify! was used to perform the remainder of the 
profiling for the C++ version of the emulator.


\subsubsection{VM Overhead}

The first data set that needs to be gathered is relationship between the time
spent in the actual emulation of the code and the overhead in creating the 
environment necessary for the emulation.  Figure~\ref{fig-sieveprofil} is a 
profile of the functions called from \verb!main! in the emulator and the 
amount of cycles spent in each of these functions and their descendents. 
The program that the emulator is running is the \verb!sieve! program, 
generating the first 3000 primes. The \verb!sieve! program is compiled with 
an optimization of O4 with gcc 2.81.

\centerfigbegin
{
\begin{tabular}{|l|r|} \hline
Functions Called from Main	& Cycles (w/descendants)  \\ \hline
executeOneInstruction	& 851,395,870,815  \\
BinaryFile::Load	& 21,582,646	\\
runDynamicLinker	& 68,794		\\
initVM			& 22,423		\\
atexit			& 398		\\
\hline
\emph{Total Cycles}	& \emph{851,417,545,076} \\
\hline
\end{tabular}
}
\centerfigend{fig-sieveprofil}{Breakdown of cycles spent in functions
called from main of the emulator - \texttt{sieve} }

From this data, it can be seen that more than 99\% of the time is spent 
executing the actual emulation code. However, sieve is a relative small 
program with a size of only 24,452 bytes and requires a relatively large 
amount of CPU cycles. A larger program that requires a smaller amount of 
CPU cycles will not fare as well.  

Figure ~\ref{fig-bannerprofile} shows the cycle breakdown of the \verb!banner! 
program displaying the word ``yo'', an example of a program that requires much 
fewer CPU cycles. The size of the \verb!banner! executable is 6,084 bytes 
compared to the size of the \verb!sieve! executable which is 24,548 bytes.

\centerfigbegin
{
\begin{tabular}{|l|r|} \hline
Functions Called from Main	& Cycles (w/descendants)  \\ \hline
executeOneInstruction	& 6,515,532  \\
BinaryFile::Load	& 2,350,321	\\
runDynamicLinker	& 70,200		\\
initVM			& 22,821		\\
atexit			& 398		\\
\hline
\emph{Total Cycles}	& \emph{8,959,272} \\
\hline
\end{tabular}
}
\centerfigend{fig-bannerprofile}{Breakdown of cycles spent in functions 
	called from main of the emulator - \texttt{banner} }

With the analysis of the \verb!banner! program, we see that only 73\% of the 
time is spent executing the actual emulation code. However, much of this time 
is spent in loading the executable which is unavoidable as even a natively 
executing program must spend time loading itself into memory. However, the 
efficiency of the loader compared to the native OS loader is currently 
unknown and requires further analysis.

A reasonable conclusion can be made that, although the efficiency of the
emulator's binary loader is not known, it cannot account for much of the
overall performance slowdown of the emulator.


\subsubsection{Child Functions Breakdown}

In order to further analyse the performance of the emulator, a list of the 
most time consuming functions were generated to see where the emulator is 
spending most of its time. Figure~\ref{fig-tenfunctions} is a list of the ten 
most time consuming functions as seen from running the sieve3000 program.

\centerfigbegin
{
\begin{tabular}{|l|r|r|} \hline
Functions		& \% Time	& \# of times called\\ \hline
execute			& 40.16  	& 6,010,495,867 \\
executeOneInstruction	& 16.86 	& 6,010,495,867 \\
executeSUBCC		& 10.57		& 737,516,466  \\
decodereg\_or\_imm	& 8.62		& 3,139,748,631 \\
executeADDCC		& 5.09 		& 342,876,189 \\
decodeeaddr		& 2.86 		& 660,334,656 \\
executeRESTORE		& 2.39 		& 94,237,151\\
executeSAVE		& 2.32 		& 94,237,151\\
executeORCC		& 2.10  	& 383,153,136\\ 
executeOR		& 1.12 		& 565,495,820\\
executeBL		& 0.94 		& 401,856,672 \\
\hline
\end{tabular}
}
\centerfigend{fig-tenfunctions}{Ten most time consuming functions of the 
	emulator when running the sieve3000 program}

A few noticeable patterns can be seen from the function breakdown in 
Figure~\ref{fig-tenfunctions}.

\begin{enumerate}
\item Most of the time is spent in the \verb!execute! function. This
function is used to match the instruction bit patterns to the assembly
equivalent for the source machine. For every instruction that needs to be
executed, one iteration of the execute function is required.

Since the matching of the bit patterns requires significant amount of
branches and also because of the frequency of this function call, it is
not surprising that this function takes a significant amount of the
processing time.

\item The function \verb!executeOneInstruction! also required a
significant amount of time. Again, this is because this function is called
once for every instruction that must be executed.

\item The amount of function calls of the top ten functions alone is
staggering as the same functions are called over and over again. Since
each function itself requires very little time to execute, all of these
functions are good potential targets for inlining.

\item Functions that require the manipulation of conditions codes such as
\verb!executeSUBCC!, \verb!executeADDCC!, and \verb!executeORCC! require a
significant amount of execution time even though the frequency of these
instructions are relatively low for the \verb!SPARC! architecture. This
indicates that each of these instructions require a significant more time
to execute than their counterparts that do not require condition codes.

\item The two decoding functions \verb!decodereg_or_imm! and
\verb!decodeaddr! both are called a significant amount of time and take up
more than 10\% of the total execution time. These decode the compound
matchining statments \verb!reg_or_imm!  and \verb!eaddr! from the SPARC
spec.

\item The two functions \verb!executeRESTORE! and \verb!executeSAVE! are
called very infrequently as they are only needed on function calls and
returns in the original source program. However, they both take up a
significant amount of time which indicates that they both are very time
consuming functions and perhaps an area that can be optimized.

\end{enumerate}


\subsection{Performance Experiments}

\subsubsection{Inlining Functions}
In order to reduce the amount of function calls, all the \verb!execute*!
and \verb!decode*! functions were inlined. This will in effect cause the
bulk of the emulator to be compiled and executed and one large function.
The benefits of this is the elimination of saving and restoring registers
and also allows the compiler to perform greater amount of optimizations.

\centerfigbegin
{
\begin{tabular}{|l|r|r|r|} \hline
Program Executed	& Original Emulator 	& Modified Emulator & \% Time Saved \\ \hline
SPEC95 130.li		& 3h 22m 49s 	& 2h 47m 52s 	& 17\% \\
sieve-3000 		& 19m 29s	& 16m 45s 	& 14\% \\
fibonacci-32		& 14.4s 	& 13.2s 	& 8\%  \\
\hline
\end{tabular}
}
\centerfigend{fig-inlineperf}{Performance improvements due to inlining of
the execute and decode* functions. Test performed on a 4 CPU Sun Ultra-80,
with low load}

As can be seen from Figure~\ref{fig-inlineperf}, the effects of inlining
is a noticeable increase in performance in the range of 10\%.


\subsubsection*{Bitwise Operation Simplification}

To address the issue of time consuming condition code manipulating
functions, much of the condition code addressing areas have been analysed
and redundancies have been removed. For example, many of the condition
code manipulation required accessing a particular bit in a variable. The
majority of this bit access is to the same bit. However, in order to
access this bit, a very cumbersome but generic operation is used.

\begin{verbatim}
    #define BITSLICE(x,lo,hi) (((x) & ((1LL<<(hi+1))-1))>>lo)
\end{verbatim}

where hi and lo are both 31. This was replaced with the following 

\begin{verbatim}
    #define BITPICK(x,lo,hi) (((uint32)x) >> 31)
\end{verbatim}

However, later analyse of the assembly code shows that the compiler when
set at a reasonable level of optimization will already perform this type
of conversion.


\subsubsection*{Register Window Modifications}

The \verb!executeSAVE! and \verb!executeRESTORE! operations contributed a
significant amount of execution time yet were called only a small amount
of times. This can be due to the register window implementation of the
emulator where there only exist one window. Therefore, every save and
restore operation will require spilling out and reading from the stack,
which results in a significant amount of memory operations.

A true sliding register window would require far less memory operations.
However, due to the SSL and SLED based nature of the emulator, a true
sliding register window implementation can not be easily accomplished.
Currently, the global registers are stored in the same array as the window
registers. Therefore, attempts to allocate a multiple window array and
simply slide the offsets of the array would cause references to the global
registers to be incorrect.

Logic can be added to every register operation to determine whether the
register in question is a global register. However, the performance hit of
this was considered to be too significant and further experimentation in
an overlapping sliding register windows was not investigated.

A non-overlapping sliding register windows implementation was implemented
as the work required is significantly less. In this implementation, the
\verb!in! and \verb!out! registers are not overlapped. The window actually
slides by 32 registers and the values of the global and the in/out
registers are copied to the new position. Therefore, all registers are in
the correct position and no additional logic is required to determine
whether a register is global.

However, in performance evaluation, the performance had actually decreased
by roughly \emph{10\%}. The reason is due to the signal handlers, which
save and restore the CPU context at every signal. When this occurs, the
whole register window is saved, and since the register window is now
signficantly bigger, the context save and restore time is significantly
slower and therefore there exists a performance decrease with this
implementation.


\subsubsection{Java Emulator Analysis}

The current performance of the Java-based emulator is approximately 15
times slower than the C++ version of the emulator\footnote{From previous
performance evaluation by Nathan Keynes}. In order to investigate the
reason for the additional performance slow down, profiling of the Java
emulator needs to be done.

The Java version of the emulator was original written with a C++ front end
that loads a \verb!JVM! and then calls the Java code. In order to profile
the Java code, the emulator must be modified to start as a Java program.

Re-implementing the emulator only required approximately one working day
and allowed the use of a Java profiler on the code. It is also a cleaner
implementation as it only uses \verb!JNI! to use C++ binary loading
libraries instead of relying on C++ code much more as in the original
version.

Performance comparison shows that this new version performed comparably to
the original version.

The profiler built into the JDK 1.4 was used to analyse the emulator.
However, the profiler was not initially working correctly as no data was
produced by the profiler.

After investigation, it was found that the reason behind the profiling
problems was due to an incorrect interpretation of the system call
\verb!exit!. Normally, the emulator will trap the exit system call and
then will perform an exit. Unfortunately, this action will also kill the
profiler process before it has a chance to output the data it gathered.

To remedy this, the emulator will perform a Java \verb!System.exit! call
instead of the normal exit system call. This will gracefully shut down the
profiler and allow the profiler to operate correctly.

\begin{verbatim}
#ifdef JAVA
    cls = env->FindClass("java/lang/System");
    func = env->GetStaticMethodID( cls, "exit", "(I)V");
    env->CallStaticVoidMethod(cls, func, o0);
    break;
#else
    exit( o0 ); err = 0; break;
\end{verbatim}

The above is the code segment inside the system call handler that was
changed in order to exit the Java version of the emulator correctly.


\subsubsection*{Performance of JNI vs. Unsafe}

A \verb!JNI! version of the emulator was developed that did not require
the use of \verb!Unsafe! classes. The \verb!Unsafe! classes, introduced in
JDK 1.4, were used to allow the Java emulator to directly access memory
addresses and also make type cast that Java would normally not allow.
Creating a JNI version of the emulator allowed comparison in performance
between the \verb!Unsafe! classes to the JNI equivalent.

\centerfigbegin
{
\begin{tabular}{|l|r|r|c|} \hline
Program Executed	& Unsafe-based 	& JNI-based & \% Extra Time JNI requires\\ \hline
sieve-3000 		& 288m 12s	& 333m 4s 	& 16\% \\
fibonacci-32		& 3m 42s 	& 4m 28s 	& 21\%  \\
\hline
\end{tabular}
}
\centerfigend{fig-UnsafeandJNI}{Performance difference of JNI and Unsafe
based versions of emulator. Test performed on a 4 CPU Sun Ultra-80, with
low load}

This data clearly show the overhead introduced by using JNI over using
Unsafe. The advantage of the JNI version is that it does not require the
use of JDK 1.4, which is still in beta testing.


\subsubsection*{Profile of Java Emulator}

Using the Java profiler, the following data was gathered from executing
the \verb!sieve! program through the Java version of the emulator. The
data gathered was very similar to the ones gathered from the C++ version
of the emulator. The methods \verb!execute! and
\verb!executeOneInstruction! still dominate the total time spent by the
emulator.

\begin{verbatim}
rank   self  accum   count trace method
   1 27.64% 27.64% 1432686509	133 sparcemu.execute
   2 20.15% 47.79% 1432686509	 31 sparcemu.executeOneInstruction
   3  9.46% 57.25% 1432686510	 98 sparcemu.getMemint
   4  6.90% 64.15% 3421444300	 36 sparcemu.decodereg_or_imm
   5  5.60% 69.75% 3421444300	 52 sparcemu.getMemint
   6  5.35% 75.11% 1432686510	 46 sun.misc.Unsafe.getInt
   7  2.98% 78.09% 3421444300	110 sun.misc.Unsafe.getInt
   8  2.50% 80.59% 1507734608	 56 sparcemu.setMemint
   9  2.41% 83.00% 1507734608	149 sparcemu.getMemint
  10  1.80% 84.80% 737504371 	 34 sparcemu.executeSUBCC
  11  1.80% 86.61% 848237913	 153 sparcemu.executeOR
  12  1.60% 88.21% 1507734608	 26 sun.misc.Unsafe.putInt
  13  1.43% 89.63% 94233413 	  65 sparcemu.executeRESTORE
  14  1.39% 91.02% 94233413 	 138 sparcemu.executeSAVE
  15  1.30% 92.33% 1507734608	 42 sun.misc.Unsafe.getInt
  16  0.93% 93.25% 401879787	 156 sparcemu.executeSRL
  17  0.85% 94.10% 376998205	  20 sparcemu.executeSETHI
  18  0.83% 94.93% 383153738	 127 sparcemu.executeORCC
  19  0.55% 95.48% 401864307	  27 sparcemu.executeBL
  20  0.46% 95.95% 342881000	  78 sparcemu.executeADDCC
  21  0.45% 96.40% 200981367	 145 sparcemu.executeSLL
  22  0.39% 96.79% 189104847	  49 sparcemu.decodeeaddr
  23  0.38% 97.18% 188499799	  23 sparcemu.executeJMPL
  24  0.37% 97.55% 282667911	 121 sparcemu.executeBCS
  25  0.32% 97.87% 189104847	  96 sparcemu.getMemint
  26  0.29% 98.16% 188499257	  43 sparcemu.executeCALL
  27  0.28% 98.44% 223767090	   7 sparcemu.executeADD
  28  0.26% 98.70% 188445153  151 sparcemu.executeBLA
  29  0.24% 98.95% 194702191	  61 sparcemu.executeBA
  30  0.22% 99.17% 94282993  	 66 sparcemu.executeSUB
  31  0.17% 99.34% 189104847	 118 sun.misc.Unsafe.getInt
  32  0.14% 99.48% 100492772 	 79 sparcemu.executeBLEU
  33  0.13% 99.61% 94242481   137 sparcemu.executeBNE
  34  0.13% 99.75% 100487320  120 sparcemu.executeBGE
  35  0.13% 99.87% 94280252    41 sparcemu.executeBEA
  36  0.12% 100.00% 94321671	  69 sparcemu.executeBE
CPU TIME (ms) END
\end{verbatim}

An interesting discrepancy between the Java and C++ versions of the
emulator is the amount of times the \verb!execute! and
\verb!executeOneInstruction! functions are called. However, that is
probably due to the difference in profilers only and is not alone enough
to discount the accuracy of the data gathered.

From the profiling results, it can be seen that the Java version of the
emulator has the additional overhead of using functions such as
\verb!getMemint!, \verb!Unsafe.getInt!, and \verb!Unsafe.putInt!. These
functions in total account for more than \emph{31\%} of the total time
spent.

Although 31\% extra overhead is significant, it does not account for the
approximate \emph{15} times slowdown of the Java emulator when compared to
the C++ emulator.


\subsubsection*{Dynamic Behaviour of Java VM}

An area that can perhaps account for the \emph{15} times slowdown is the
dynamic behaviour of the Java virtual machine. If the Java VM
(\texttt{Hotspot 1.4}) was able to discover all the hot paths and make
them into compiled code, then the performance of the Java version should
be very close to a natively compiled C++ version. Further more, the Java
VM could even be smart enough to recognize the nature of the code and
realize that most of the functions are good candidates for inlining
further improving performance. The following is a breakdown of the dynamic
behaviour of the Java VM.

\begin{verbatim}
Flat profile of 17670.26 secs (879033 total ticks): main

  Interpreted + native   Method                        
 75.7% 663823  +  1335   sparcemu.execute
  0.0%     0  +     7    emumain.main
  0.0%     0  +     2    sparcemu.doTrap
  0.0%     0  +     1    sparcemu.decodereg_or_imm
  0.0%     0  +     1    java.util.zip.ZipFile.getEntry
  0.0%     0  +     1    java.lang.String.charAt
  0.0%     1  +     0    sparcemu.executeBL
  0.0%     1  +     0    sparcemu.executeSLL
  0.0%     1  +     0    sparcemu.executeADDCC
  0.0%     0  +     1    java.util.zip.ZipFile.open
  0.0%     1  +     0    sparcemu.setMemdouble
  0.0%     1  +     0    sparcemu.executeBGU
  0.0%     0  +     1    java.lang.Shutdown.halt
  0.0%     1  +     0    sparcemu.executeSAVE
  0.0%     0  +     1    java.io.FilePermission.newPermissionCollection
  0.0%     1  +     0    sparcemu.executeRESTORE
  0.0%     1  +     0    sparcemu.getMemdouble
 75.7% 663831  +  1350   Total interpreted

     Compiled + native   Method                        
  4.6% 40134  +     0    sparcemu.getMemint
  3.1% 27390  +     0    sparcemu.executeOneInstruction
  2.8% 24458  +     0    sparcemu.decodereg_or_imm
  1.6% 14220  +     0    sparcemu.executeSUBCC
  1.5% 12769  +     0    sparcemu.run
  0.8%  7368  +     0    sparcemu.executeRESTORE
  0.8%  6753  +     0    sparcemu.executeADDCC
  0.7%  6113  +     0    sparcemu.executeSAVE
  0.6%  5025  +     0    sparcemu.setMemint
  0.6%  5001  +     0    sparcemu.executeOR
  0.5%  4444  +     0    sparcemu.executeORCC
  0.3%  2470  +     0    sparcemu.executeSETHI
  0.3%  2352  +     0    sparcemu.executeSRL
  0.3%  2250  +     0    sparcemu.executeBL
  0.2%  2007  +     0    sparcemu.decodeeaddr
  0.2%  1806  +     0    sparcemu.executeBCS
  0.2%  1708  +     0    sparcemu.executeBLA
  0.2%  1460  +     0    sparcemu.executeADD
  0.1%  1198  +     0    sparcemu.executeCALL
  0.1%  1146  +     0    sparcemu.executeJMPL
  0.1%  1084  +     0    sparcemu.executeSLL
  0.1%  1030  +     0    sparcemu.executeBA
  0.1%   908  +     0    sparcemu.executeBNE
  0.1%   891  +     0    sparcemu.executeBEA
  0.1%   693  +     0    sparcemu.executeSUB
 20.1% 176577  +     3   Total compiled (including elided)

         Stub + native   Method                        
  0.0%     0  +    12    sparcemu.doTrap
  0.0%     0  +    12    Total stub

  Runtime stub + native  Method                        
  2.0% 17440  +     0    interpreter_entries
  2.0% 17440  +     0    Total runtime stubs

  Thread-local ticks:
  0.0%     1             Blocked (of total)
  0.0%     2             Class loader
  2.2% 19706             Interpreter
  0.0%     9             Compilation
  0.0%    68             Unknown: running frame
  0.0%     1             Unknown: calling frame
  0.0%     1             Unknown: no last frame
  0.0%    32             Unknown: thread_state


Global summary of 17670.26 seconds:
100.0% 879033            Received ticks
  0.0%     8             Compilation
  0.0%     2             Class loader
  2.2% 19706             Interpreter
  0.0%   102             Unknown code

\end{verbatim}

The above data shows that the majority of the time was spent interpreting
code instead of executing compiled code. The function
\texttt{sparcemu.execute}, which when interpreted, accounts for 75.7\% of
the execution time. However, it was not found to be hot by the VM and thus
was not compiled. This interpretation probably accounts for a significant
portion of the slowdown of the Java version of the emulator compared to
the C++ version. Also, many of the functions that the VM decided to
compile such as \texttt{sparcemu.executeSLL} and
\texttt{sparcemu.executeBEA} are really not executed that frequently and
the compilation overhead for these functions may not be justified. Further
investigation with fully compiled Java code would be useful to prove or
disprove these theories.







		% interpreter generation

	\chapter{Instrumentation of an Interpreter via Specifications}
\label{ch-genemu-instr}

{\small
\begin{flushright}
Design: Cristina, David; Implementation: David; 
Documentation: David Ung [May 2001], Cristina [Jan 2002]
\end{flushright}
}

This chapter describes the specification file format used to automatically 
add instrumentation code to the \walk\ based emulator (described in 
Chapter~\ref{ch-genemu}). 
Example instrumentation files are also given to demonstrate the type of 
instrumentation that it can create and how it is integrated into the emulator.


\section{Instrumentation}
The goal of this research is to determine an inexpensive and easy to use way 
to add instrumentation to the emulator.  The type of instrumentation to be 
added describes ways to identify what sections of the source program are hot 
(i.e. frequently executed), so that code generation can be done to improve 
the overall performance of the execution.

Existing tools such as EEL~\cite{Laru95} and ATOM~\cite{Eust95} provide an 
interface to add code by specifying the level where instrumentation should 
take place and what to instrument.  
Such tools reconstructs the application to be instrumented 
to an intermediate representation in memory and then modifies its structure 
(through CFGs and basic blocks) to add the instrumentation into the application.
Finally, the binary is rebuilt and an instrumented version is emitted.

There are two fundamental problems with the approach described above when 
trying to add instrumentation to the \walk\ emulator:

\begin{enumerate}
\item When the emulator is running, it is emulating the behaviour of a program. 
The execution of the program determines which paths to take during runtime and 
hence indirectly affects which part of the emulator is invoked.  Adding 
instrumentation to the emulator will instrument the emulator.  
Although this can indirectly give information about the runtime behaviour of 
the source program, the information gathered will be more in the scope of the 
emulator, thus losing emphasis on the source program.  
In particular to the attempt to find hot traces, instrumentating the emulator 
may tell us that the function \texttt{executeBNE()} is hot, but the 
information about which instruction cause this in the source program is 
not revealed.  
It is possible to obtain information about the source program through the 
instrumenting the emulator, but the task is not an easy one.  It requires 
knowledge about the internal workings of the emulator and makes the works of 
instrumentation difficult to use.

\item The emulator is a very low level data processor.  Instead of multiple 
level of abstractions found in existing tools, the only abstraction of the 
emulator is at the instruction level.  
This low level abstraction greatly limits the amount of calls that can be 
made to predefined functions provided by high level instrumentation tools.  
For example, high level functions such as FOREACH\_EDGE() and FOREACH\_BB().
\end{enumerate}

To provide a flexible and powerful instrumentation at the instruction level, 
the emulator itself should provide the instrumentation.  Since the emulator 
is automatically generated, this motivates the idea of automatically adding 
instrumentation code as part of the emulator. 
Being part of the emulator code, the instrumentation has access to variable 
and locations internal to the emulator.  This approach allows direct control 
over what to instrument.  The goal is to instrument the source program, not the 
emulator. 


\subsection{Existing instrumentation tools}
The use of instrumentation provides opportunities in binary editing, emulation,
observation, program comprehension and optimization.  Although many tools exist
that can modify binaries, the implementation of the actual code modification 
are typically fused in detail with the application or executable itself.  
But several tools exist that provide a high level interface (typically 
through a set of libraries) to easily access its instrumentation facilities.  
The following are some examples of such tools:

\begin{enumerate}
\item Srivastava and Wall's OM system~\cite{Dec94}, a library for binary 
	modification.  It requires relocation from object files to analyse 
	control structure and to relocate edited code.
\item ATOM~\cite{Eust95} provide an interface to the OM system.  
	Very high level of abstraction, simplifies the writing of tools.
\item QPT~\cite{Laru94} by Larus and Ball, a profiling and tracing tool.
\item EEL~\cite{Laru95}, also a library for building tools to analyze and 
	modify binaries.
\end{enumerate}

Both EEL and ATOM are large libraries that provide a rich set of routines for 
instrumentation. 
Different levels of the abstraction in these tools allows control over what 
level the tool wants to instrument through calls to those library rountines.  
The following example EEL code shows the use of the library:

\begin{verbatim}
executable* exec = new executable("test_program");
exec->read_contents();

rountine* r;
FOREACH_ROUTINE (r, exec->rountines()) {
  cfg* g = r->get_control_graph();
  bb* b;
  FOREACH_BB(b, g->blocks()) {
    edge* e;
    FOREACH_edge(e, b->succ()) {
      count_branch(e);
    } 
  }
}
\end{verbatim}

The types \texttt{executable}, \texttt{rountine}, \texttt{cfg}, \texttt{bb} and 
\texttt{edge} are data structures provided by the library for different level of
abstraction of the binary.  The tools simply make calls to library routines 
such as \texttt{read\_contents()}, \texttt{FOREACH\_BB} and 
\texttt{get\_control\_graph()}.  The only function that 
needs to be written by the tool builder is \texttt{count\_branch()}.
The concept in ATOM is similar to EEL in that the tool builder make uses of 
library rountine to access the different levels of abstraction in the binary.


\section{Instrumentation specification}

The instrumentation code is written in a separate specification file that is 
linked as part of the emulator at the time of generating the New Jersey 
Machine Code (NJMC) matching file.
Code can be added at the instruction level under the \texttt{DEFINITION} 
section.  
The list of instructions that you wish to act on is specified as a table.  
Instrumentation code is then specified for a table by adding relevant code with
respect to the main body of their emulation rountines.
For example, if you want to count the number of times a particular set of 
branches is taken in an X86 program, the instructions to be monitored are 
specified in a table as follows:

\begin{verbatim}
DEFINITION

jump32s	[ "JVA", "JVNBE", "JVAE", "JVNB", "JVB", "JVNAE", "JVBE",
          "JVNA", "JVC", "JVCXZ", "JVE", "JVZ", "JVG", "JVNLE", "JVGE",
          "JVNL", "JVL", "JVNGE", "JVLE", "JVNG", "JVNC", "JVNE", "JVNZ",
          "JVNO", "JVNP", "JVPO", "JVNS", "JVO", "JVP", "JVPE", "JVS", 
          "JMPJVOD" ]
\end{verbatim}

In order to count occurrences of the branch instructions listed
in the \texttt{jump32s} table, the semantics of the branch instructions is 
extended to increment a counter.  This is expressed in the INSTR language
as follows: 

\begin{verbatim}
jump32s label
{
    increment_counter(SSL(%pc), PARAM(label));
    SSL_INST_SEMANTICS
}
\end{verbatim}

where the function \texttt{increment\_counter} is defined in the 
\texttt{IMPLEMENTATION\_ROUTINES} section of the specification file, 
and \texttt{SSL\_INST\_SEMANTICS} refers to the semantics of the instruction 
as specified in the semantic description file SSL.  
For illustration purposes, we show the section 
\texttt{IMPLEMENTATION\_ROUTINES}, where the function 
\texttt{increment\_counter} is implemented. 
  
\begin{smallverbatim}
IMPLEMENTATION_ROUTINES

#include <map>
#include <iostream>

// map edge to execution counts
map< pair<unsigned, unsigned>, int > edge_cnt;

// increments the branch count for edge (addr1, addr2)
void increment_counter(int addr1, int addr2) {
    // construct the edge.
    pair<unsigned, unsigned> edge = 
            pair<unsigned, unsigned>(addr1, addr2);
    map< pair<unsigned, unsigned>, int >::iterator i;
    if ((i = edge_cnt.find(edge)) == edge_cnt.end()) 
    {
        // not found in map, add it and set count to 1
        backedge_cnt[edge] = 1;
    } else 
    {
        (*i).second ++;      // increment counter by 1
    }
}
\end{smallverbatim}


To build the instrumented SPARC emulator, the SLED, SSL and the 
instrumentation file (INSTR) are included as part of the build to generate 
the matching \texttt{.m} file:

\begin{verbatim}
tools/genemu -i sparc.backbranches.inst machine/sparc/sparc-core.spec 
    machine/sparc/sparc.ssl sparcemu.m
\end{verbatim}

The \texttt{-i} option includes the instrumentation file 
\texttt{sparc.backbranches.inst} into the build of the SPARC emulator.
The contents of \texttt{sparc.backbranches.cnt500.inst} is found in the 
Appendix to this Chapter.


\subsection{Syntax of the INSTR Language}

The instrumentation file consists of two main sections:
\begin{enumerate}
\item Definition, 
\item Fetch-execute cycle, and 
\item Support code.
\end{enumerate}

The definition section specifies which instructions are to be instrumented and 
their corresponding instrumentation code.  
The fetch-execute cycle section specifies what, if any, commands need to 
be executed at each iteration of the loop. 
The support code section contains additional code that the program may call 
as part of instrumentation.  This code is expressed in the C language.  
The EBNF for the language is: 

\begin{verbatim}

specification:      parts+

parts:              definition | support_code        

definition:	        DEFINITION instrm+

instrm:	            table | semantics

table:              STRING [ SLED_names ]

semantics:          (STRING parameter_list instrument_code)+ 
                    (FETCHEXECUTE instrument_code)*

parameter_list:	    STRING (, STRING)*

instrument_code:    { (action)* SSL_INSTR_SEM }

support_code:       IMPLEMENTATION_ROUTINES c_code

\end{verbatim}

where \texttt{action} contains any valid C/C++ code written by the user.  
This piece of code is attached to the instructions specified in the table 
along with the instruction's semantics.
The following special symbols may be inserted into \texttt{action}:

\begin{enumerate}
\item \texttt{SSL\_INST\_SEMANTICS}: stands for the semantics of the 
	instruction, as described in the SSL specification file, 
\item \texttt{PARAM({string})}: indicates the value of the 
	operand \texttt{{string}} of the current instruction.  E.g.
	\texttt{PARAM(label)} of the \texttt{BA} instruction is the 
	instruction's first operand.
\item \texttt{SSL(\%{register name})}: indicates one of the machine 
	registers specified in the SSL file. E.g. \texttt{SSL(\%pc)} is the 
	location holding the value of the emulated PC register.
\end{enumerate}


\section{Appendix}

\subsection{Listing of an Instrumentation File}

\begin{smallverbatim}
# File: sparc.backbranches.cnt50.inst
# Desc: This file contains the list of instructions and actions 
#       used for instrumentation.
#       The file instruments backbranches and invokes the trace 
#       builder to build hot traces used for optimisation.

DEFINITION

branch [ "BA", "BN", "BNE", "BE", "BG", "BLE", "BGE", "BL",
         "BGU", "BLEU", "BCC", "BCS", "BPOS", "BNEG", "BVC", "BVS",
         "BNEA", "BEA", "BGA", "BLEA", "BGEA",
         "BLA", "BGUA", "BLEUA", "BCCA", "BCSA", "BPOSA",
         "BNEGA", "BVCA", "BVSA" ]	
branch label 
{
    int oldpc = SSL(%pc)		
    SSL_INST_SEMANTICS
  
    // check back branches
    if (oldpc > PARAM(label)) {
        if (trace_mode) {
            end_build_trace();
		} else {		
            // branch is taken
            if (SSL(%npc) == PARAM(label)) {			
                increment_counter(oldpc, PARAM(label));
            }
        }
    }
}

FETCHEXECUTE
{
    if (trace_mode) {
        add_to_trace(SSL(%pc));		
    }
    SSL_INST_SEMANTICS
}

INSTRUMENTATION_ROUTINES
#include "emu/backbranches.cnt500.cc"
\end{smallverbatim}


\subsection{Listing of Support Code: backbranches.cnt500.cc}

\begin{smallverbatim}
#include <map>
#include <list>
#include <iostream>

// an edge is made up by a pair of addresses (branch_inst, target_inst)

// the upper value that an edge must reach to trigger the building of traces.
// this is value is incremented by 500 each time start_trace is called.
int trigger = 500;

// map edge to execution counts
map< pair<unsigned, unsigned>, int > backedge_cnt;

// informs the emulator whethere it is in trace mode.  If so, 
// it will add the current instruction at %pc to the trace.
int trace_mode = 0;

// holds the list of instructions in the current trace
list<unsigned> trace;	


// ***************************************************************************
// Function definitions
// ***************************************************************************

// prints the trace to stderr and exit		
void print_trace() {
    cerr << "Trace found: " << endl;
    list<unsigned>::iterator i;
    for (i = trace.begin(); i != trace.end(); i++) {	
        cerr << hex << (*i) << endl;
    }
    abort();
}

// signal end of trace and invoke optimizer
// at the moment it just prints the trace to the screen
void end_build_trace() {
    trace_mode = 0;
    print_trace();
}
		
// add instruction curr_inst to the trace list
// if curr_inst is the back branch that started trace building, 
// then trace building
void add_to_trace(unsigned curr_inst) {
    trace.push_back(curr_inst);
    // check to see if it is the end of trace
    if (trace.front() == curr_inst) {
        end_build_trace();
    }
}

// start building a hot trace from back branch edge
void start_trace(pair<unsigned, unsigned> edge) {
    cerr << "Starting trace finding!" << endl;	
    trace_mode = 1;
    // increase the trigger count or interate through the entire 
    // map and reset all counters or just empty the map
    trigger += 500;
    trace.clear();
    // add edge to the current trace
    trace.push_back(edge.first);
}

// increments the branch count for edge (addr1, addr2)
// if count reaches trigger, call start_trace to begin trace building.
void increment_counter(int addr1, int addr2) {
    // construct the edge.
    pair<unsigned, unsigned> edge = 
                       pair<unsigned, unsigned>(addr1, addr2);
    map< pair<unsigned, unsigned>, int >::iterator i;
    if ((i = backedge_cnt.find(edge)) == backedge_cnt.end()) {
        // not found in map, add it and set count to 1
        backedge_cnt[edge] = 1;
    } else {
        (*i).second ++;              // increment counter by 1
        if ((*i).second >= trigger) {
            start_trace(edge);
        }
    }
}
\end{smallverbatim}


\subsection{Sample Output}

Given the specification file in this Appendix, the newly generated 
\texttt{.m} file will now contain the extra instrumentation code as part 
of the emulation functions.  For example, the code for the branch always, 
\texttt{BA}, and the branch not equal, \texttt{BNE} instructions is: 

\begin{smallverbatim}
void executeBA( sint32_t reloc )
{
  int oldpc = regs.r_pc;
  
  regs.r_npc = (((1) == (0)) ? regs.r_npc : reloc);

  // check back branches
  if (oldpc > reloc) {
    if (trace_mode) {
        end_build_trace();
    } else {		
        if (regs.r_npc == reloc) {			
            increment_counter(oldpc, reloc);
        }
    }
  }
}

void executeBNE( sint32_t reloc )
{
  int oldpc = regs.r_pc;

  regs.r_npc = (((regs.r_ZF) != (0)) ? regs.r_npc : reloc);

  // check back branches
  if (oldpc > reloc) {
    if (trace_mode) {
        end_build_trace();
    } else {		
        if (regs.r_npc == reloc) {			
            increment_counter(oldpc, reloc);
        }
    }
  }
}

...

void executeOneInstruction()
{
    if (trace_mode) {
        add_to_trace(regs.r_pc);		
    }

    sint32_t tmp;
    tmp = regs.r_pc;
    regs.r_pc = regs.r_npc;
    regs.r_npc = ((regs.r_npc) + (4));
    regs.rd[0] = 0;
    execute(tmp);
}
\end{smallverbatim}


 		% instrumentation specification

	\chapter{PathFinder}
\label{ch-pathfinder}

{\small
\begin{flushright}
Design: David, Cristina; Implementation: David; 
Documentation: David Ung [Aug 2001], Cristina [Jan 2001]
\end{flushright}
}

This file describes the internals of PathFinder; a SPARC to ULTRASPARC code 
execution system.


\section{Architecture Overview}

\psfigbegin{figures/pathfinder.eps}{8cm}
\psfigend{fig-pathfinder2001}{PathFinder: The Implementation of the 2001
        Walkabout Framework}

The PathFinder's architecture is illustrated in Figure~\ref{fig-pathfinder2001}.
The SPARC interpreter is that automatically generated by \texttt{genemu\_i}
when using an instrumentation file that determines hot paths based on a 
given set of termination conditions.
We instrumented branches to count how many times a given branch was 
executed; when the counter reached a trigger, a trace of the hot 
path could be generated.

One of the instrumentation modes replicated as close as possible 
Dynamo's~\cite{Bala00} next executing tail (NET) method to determine
hot paths, as a way to evaluate it in comparison to other methods. 
Hence, the trace was generated during the next execution of the code 
and the trace selection termination conditions were:  

\begin{itemize}
\item A back branch is met, or 
\item The trace buffer reaches its limit size
\end{itemize}

The code generator is then called to generate code based on the trace
that was collected.  
During tracing, various data is also collected. (memory references of the
indirect jumps, target of loads and stores, return addresses, etc) which 
are stored in a data structure called \texttt{reference\_map}, that will 
be used by the code generator.
The code generator uses the NJMC to encode assembly instructions into binaries.
It builds labels for every source address that is in the trace, these label will
be used by the fragment to determine the relative offset of instructions such 
as branches.
The trace instructions are then iterated through a second time though this time 
with each instruction, it calls \texttt{encode()} to write the instruction 
to the relocatable block (a piece of memory set aside to do encoding, which 
eventually gets copied into the fragment cache).
Code layout, inversion and other optimizations are done as part of the encoding.
At the end of trace encoding, all portals\footnote{
Portals are edges going into or out of a hot path region.  
For an edge (a,b), an in-portal is the node `a' such that node `b' belongs
to the hot path.  Similarly, an out-portal is a node `a' that is part of
the hot path.  For examples refer to~\cite{Cifu00g}. 
} that were generated are encoded at the end after the code layout 
instruction starting with the out portal of the back branch itself,  
followed by a series of other portals.
The relocatable block is then copied into the fragment cache and an entry into
the fragment is added to the map of entry points.

Execution does not immediately enter the newly generated fragment. 
The semantics of the back branch is executed before the code generator is
called, hence the delay slot instruction is the next instruction to be executed.
Control is passed back to the emulator, and another interation is done under the
emulator before the fragment is called.
When the same back branch is reached, Pathfinder does a context switch to get
ready to enter the fragments.
A context switch involves the moving of values from the virtual registers to 
the real host machine registers\footnote{
The current implementation uses some global registers as scratch
registers, though some space has been set aside in 
\texttt{host\_context.temp[16]} to hold 16 temporaries during context 
switching.  One could change the code so that those global registers are 
not overwritten.}
Other registers including integer and floating point condition code are written
to the correct fields in the host state registers.
Once control is in the fragment, no emulation or instrumentation takes place.
PathFinder gets control only when one of the exits of the fragment is taken.
The exit (out portal) will do a save to preserve integrity of the fragment
state.  
The contents of real registers and condition codes are written
back to the virtual locations.
To issue control to the emulator, one could jump directly to it.  
Instead, the Pathfinder sets up a virtual PC and a virtual nPC, and sets 
CTI to 1.  
A return from the routine will end up in the main loop of the emulator.
Fragment linking is done as part of jumping from fragment to emulator.  Some
details of the implememtation is described in Section~\ref{sec-futurework}.


\section{Relocator}
The relocator was the first attempt to use an instrumentation profile to drive
the emulator to generate traces.
Basically, the relocator instruments back branches and generates traces using 
the next executing tail scheme.
The following are design decisions with the relocator:

\begin{enumerate}
\item The address reserved by the emulator to hold the context of the source
program remains unchanged.  This memory block is the member VM of one of the
systems personalities.  
The variable \texttt{mem} is the base address of the VM.  
Each memory reference is adjusted by \texttt{mem} as part of the macros in 
the \texttt{emu.m} file.
The relocator preserve this behaviour, hence the fragment code generated need to
adjust every load and store instruction to correct the accessing of the source
program's data and text area.

\item Delay slot relocation.
If the delay slot of a control transfer instruction is a load or a store,
relocation can be difficult.  Since the relocation will involve a series of
adjustment instructions, it will result in code increase that is too big to fit
in a delay slot.  Some careful rearrangement of instructions is required for 
this case.

\item Mirrored stack.
A mirrored stack is essential for the correct execution of the relocator.  
At the time when the program is in the fragment, the host machine's stack
pointers are pointing to a virtual address space within the VM (set aside for
source program mapping, see above).  
However, physically, the VM is mapped on top of the emulator's context, hence 
the stack pointers are physically pointing inside the emulator, not the source
programs space.  This is okay, since every memory reference (including stack
references) are relocated to the VM.
The only problems with this approach is when a flush spill trap happens.  
Since it is not under the control of the emulator or the fragment (both which 
are relocated memory accesses), the trap will write the area physically pointed 
to by \texttt{\%FP} and \texttt{\%SP}.
A mirrored stack is created to combat this by mmapping a block of memory so the
spill handler can use as temporary storage.
\end{enumerate}


\section{Building}
The PathFinder is link by it's instrumentation profile to the emulator.  The
instrumentation file lives in the directory specified as part of the configure
parameter \texttt{--with-instrm=<instrm\_dir>}, where \texttt{instrm\_dir} 
will have a minimum of 2 files with the following exact names:
\begin{itemize}
\item profile.inst
\item make.rules
\end{itemize}

\texttt{profile.inst} holds the instrumentation rules that hooks to the
emulator.
See Chapter~\ref{ch-genemu-instr} for syntax of the instrumentation profile.
\texttt{make.rules} are extra rules that need to be incorporated into the 
\texttt{make} file so that the correct dependencies can be determined.
Also, any extra defines and link options are included.


\section{Future Work}
\label{sec-futurework}

Here's a list of things that can be done to improve the system. 

\subsection{Fragment Linking}
The way the fragments are linked can be improved at branch exits and 
on indirect branch exits. 

\begin{enumerate}
\item Branch exit: 
The current implementation of full fragment linking involves patching of
instructions to jump to the corresponding fragment entry.
An out portal has the form of:
\begin{verbatim}
save..
call +2
nop
read condition code
setup parameters
jump to (jump_out_fragment)
\end{verbatim}

The patching takes place by the Dispatcher (part of 
\texttt{jump\_out\_fragment}).
If the edge addresses have a corresponding fragment generated, the exit is
patched so that subsequent exits will jump directly to it.
The instructions at the out portal are patched with a branch to the 
target address; either 

\begin{verbatim}
sethi %hi(target_fragment) ....
jmpl %lo(target_fragment) ...
nop
\end{verbatim}

or 

\begin{verbatim}
ba target_fragment
nop
\end{verbatim}

To find where the out portal is, the patcher uses \texttt{\%i7}.  
Note that the portal has a call instruction, which when executed will store the
current \texttt{\%PC} into \texttt{\%o7}.
Improvement can be done instead of patching the portals, patch the branch exit
that jumps to the portal!
This will give an estimated of 20\% improvement in speed for compress95-O4.
It removes 2 instructions in the fragments and one of them is a branch.
Other benchmarks will probably benefit greatly as well, about 10\%.

\item Indirect exit: 
Indirect jumps are generated to be compared with a predicted value (profiled 
at tracing time).  
Exits from an indirection suggest another value or a more accurate value 
(the next executing tail profiled value is not correct).
An extra compare to the new value of the exit should be inserted to reflect the
changes or program behaviour.
Unfortunatly, space for the extra compare is not something an existing out
portal would have.
Currently, the PathFinder does half of fragment linking in that in does not 
patch the indirection out portal, but just jumps to it.  
The will incur the cost of context switch not every exit.
To correct this, re-encode the exit at a separate location and add the extra
compare to it.  Also, patch the exit when appropriate.
\end{enumerate}


\subsection{Optimisations for V9}
This will make the tool more like an SPARC to ULTRASPARC optimiser, 
to experiment with whether such approach is feasible in practice 
with large application programs.

\begin{enumerate}
\item Branch with predition: this was implemented, there was little or 
	no improvement in the executed code.
\item Data prefetching: mostly safe and will probably give the most improvement.
\item Conditional moves: not useful for next executing tail, but can be 
	useful for other schemes.
\item Replacing of V7 library calls with V9 floating point instructions.
\end{enumerate}


 		% sparc to sparc translation tool 

	\chapter{Debugger}
\label{ch-debugger}

{\small
\begin{flushright}
Design: Bernard Wong [Oct 2001]; Documentation: Bernard Wong [Jan 2002]
\end{flushright}
}

The experiments with the emulator have so far been very positive and show much 
promise in enabling a more robust dynamic method to perform binary translation. 
Currently, the emulator works very well with most small conventional user 
programs on the SPARC. However, large amounts of time was required to debug the 
emulator in order for it to reach this working state for the SPARC Solaris 
platform. This is because it is very difficult to debug the emulator, as it 
often requires the programmer to read pages and pages of SPARC assembly code 
to spot the one mistake the emulator makes. 

The goal of the graphical debugger for the emulator is to reduce the time and 
effort necessary in debugging the emulator for each new platform which it is to 
support. This goal drives the two main necessary requirements in the design of 
the debugger. These requirements are the following:

\begin{enumerate}
\item To allow easier and quicker debugging of the emulator, and  
\item To run and support every platform which the emulator supports.
\end{enumerate}

Please note that the emulator debugger currently depends on the the Unsafe 
package which is only available with Java 1.4 (in Beta testing at the 
time of writing).


\section{Overview of Design}

The debugger is written in the Java language and makes use of the Swing 
package. The program can be broken up into the following sections:

\begin{enumerate}
\item Graphical Section - \texttt{emuDebug.java}
\item Emulator Connection Section - \texttt{emuProcess.java}, 
		\texttt{emuLib.skel}
\item Disassembler Section - \texttt{disasm.java}
\item Debugger Preprocessor - \texttt{emuDebugGen} 
\end{enumerate}


\subsection{Graphical Section}

The graphical section, as its name implies, is a collection of classes that 
deals with the different graphical parts of the debugger. Any modification to 
GUI should be made to this section. There are currently 5 main windows to the 
debugger: Disasm Output, Command Window, Register Window, Trace Window and Misc.
Window. 

Some of the GUI components' functionality are not currently implemented. These 
include the Relocation radio buttons, the Trace Window, the Float Registers 
frame and the View at Mem Address box. The Relocation buttons and Trace Window 
are intended for use by the PathFinder, as it would give the PathFinder tool the
ability to display addresses and assembly instructions of the collected traces 
which would greatly help debugging that tool as well.


\subsection{Emulator Connection Section}

The emulator connection section is a separate Java program which interacts with 
the emulator and communicates with the rest of the debugger via sockets. Every 
time a breakpoint is reached the emulator connection section will send the 
current emulator state information to the graphical section serving as a bridge
between the debugger and the emulator. 

The main reason for separating the emulator connection section into a separate 
program from the graphical section is because of the stability of the emulator.
During many of the development stages of the emulator, it would often crash if 
it encountered certain combination of instructions. If the emulator, which is a 
native C program, crashes, it will cause the emulator connection section to also
crash as the emulator connection section is connected to the emulator via 
the Java Native Interface (JNI). 
By separating the emulator connection section from the graphical section, it 
allows the debugger to gracefully recover from the crash and collect all the 
relevant emulator state information just before the crash occurred without 
requiring a restart of the entire debugger. 


\subsection{Disassembler Section}

The disassembler section uses the existing automatically generated disassembler 
and formats the disassembled information into a format compatible with the 
Graphical Section. Since the automatically generated disassembler is generated 
by the same tool as the automatically generated emulator, any platform which the
emulator supports would also be supported by the disassembler. Therefore, using
this disassembler will help the debugger satisfy its requirement to support all
the platforms which the emulator supports.


\subsection{Debugger Preprocessor}

Finally, the debugger preprocessor is an effort to allow the debugger to easily 
support every platform which the emulator supports. Places in the debugger that 
require platform specific information are marked with special preprocessor 
symbols that are later replaced with platform specific code that is  
automatically generated via the use of specification files. A very simple parser
used to perform the replacements can be found in the \texttt{emuDebugGen} 
directory. The specification file is \texttt{machine/sparc/emuDebugSPARC.spec} 
(for SPARC machines) and contains the following section headings: 
\texttt{ConditionCodes}, \texttt{IntegerRegisters}, \texttt{ProgramCounter}, 
and \texttt{MiscRegisters}.  The specification definition is currently 
only sufficient to support SPARC instructions. Additional headings support will 
need to be added in order to support other platforms.

Performing the command

\begin{verbatim}
java CodeGen src/machine/sparc/emuDebugSPARC.spec \
     src/machine/sparc/emuDebugSPARC.m \
     src/machine/sparc/emuLib.skel
\end{verbatim}

will generate the final platform specific Java files from the skeleton and 
specification files. The included make file will already perform the necessary 
Java files generation.

\section{Current Status}
The current version of the debugger allows for the execution of the emulator to 
be controlled via the graphical panels of the debugger. The emulator can be told
to emulate one instruction at a time (stepping through emulated machine 
instructions), or can be told to stop at breakpoints specified by memory 
addresses. Breakpoints can be added graphically as the assembly instructions of 
the executing program are shown. However, the assembly instructions when the 
program enters a library file is not shown. Therefore it is best to compile the 
executable to be emulated statically during the debugging phases. At each 
breakpoint, the current integer register, control flags, PC and nPC and other 
important register values are shown. The debugger can also restart the emulator
at any point of execution.

Currently, only the SPARC emulator is supported by the debugger. In order to 
support other platforms, the specification file definition will need to be 
expanded. A more mature parser may need to be written in order to easily parse 
an expanded specification definition (the current parser is only meant to 
quickly test the feasibility of using specification files to generate Java 
files). 

Support for PathFinder is also not currently implemented. Supporting 
Pathfinder with the debugger would be a very worthwhile feature as it would 
greatly help to debug the complex PathFinder code.


		% the Java-based debugger interface

\appendix
    %
% 29 Aug 01 - Cristina: created 
% 24 Oct 01 - Brian: Added testing section. Made various edits. 
%  8 Jan 01 - Cristina: updated
%

\chapter{Building Walkabout}
\label{ch-config}

{\small
\begin{flushright}
Documentation: Cristina [Aug 01, Jan 02]
\end{flushright}
}

This chapter contains notes on how to configure the \walk\ framework
for a given platform.  
The tools that can be built within the \walk\ framework include 
an emulator, a disassembler, a pathfinder and a debugger. 
Most of the examples are for the SPARC architecture as this was 
our development platform.  


\section{Compilers and Tools Needed to build \walk}
We use gcc 2.95.3, however, we do not make use of any of the new 
classes that are not available in 2.95-2, such as sstream. 
Note that we make use of namespaces sparsely in the code and these
are not supported by the gcc 2.8.1 version of the compiler, 
but they are in egcs-1.1.2 (gcc 2.91.66). 

For debugging, gdb 5.0 works well with gcc 2.95.3.


\subsection{Special tools needed to build \walk}

\walk\ has many source files that are generated from other source files,
or from specifications. It is possible to make \walk\ without installing
these tools, but if you want to make significant changes to \walk, you will
need those tools.

To make \walk\ without the special tools, use the \texttt{--enable-remote}
configuration script (see above).

The special tools are as follows.

\begin{itemize}
\item The New Jersey Machine Code Toolkit, ML version. This tool reads machine
specifications, and in association with a matcher (\texttt{.m}) file, generates
binary decoders. For details and downloading, see
\texttt{http://www.eecs.harvard.edu/~nr/toolkit/ml.html}.
\item Bison++ and Flex++, C++ versions. Note that the GNU tool bison++ is
{\it not} suitable; \walk\ needs the special versions from France, which are
C++ aware. If you get lots of errors from running bison++, you have probably
got the wrong version! Download these tools from
\texttt{ ftp://ftp.th-darmstadt.de/pub/programming/languages/C++/tools/flex++bison++/LATEST/}
You might also try downloading these tools from one of the various mirror sites such as
\texttt{http://sunsite.bilkent.edu.tr/pub/languages/c++/tools/flex++bison++/LATEST/}.
To test if you have the correct version, you should get results similar to:
\begin{verbatim}
%  bison++ --version
bison++ Version 1.21-7, adapted from GNU bison by coetmeur@icdc.fr
\end{verbatim}
If searching the web for these tooks, include the author's name ("coetmeur")
as a keyword.
\item The Tcl shell (\texttt{tclsh}). This tool is only needed to run the
regression test script (\texttt{test/regression.test}).
\texttt{tclsh} and the \texttt{tcltest} package are part of
Tcl/Tk releases 8.0 and newer.
You may well find that these are already installed on
your Linux or other system. Otherwise, see web pages such as
\texttt{http://www.sco.com/Technology/tcl/Tcl.html}.
\end{itemize}


\section{Configuration Notes}
In order to instantiate a translator out of the \walk\ framework, 
you need to configure \walk\ to run on your host machine by instantiating 
a set of source and target machines.  Figure~\ref{fig-mach-names} 
lists the names used within \walk\ to describe machine specifications, and 
the associated instruction set version that is specified.  

\centerfigbegin
\begin{tabular}{|l|l|} \hline
Name	& Description \\ \hline
sparc	& SPARC V8 (integers and floats) \\
pent	& 80386 (integers and floats) \\ \hline
\end{tabular}
\centerfigend{fig-mach-names}{Names of Machines and Versions Supported 
	by the \walk\ Framework}

You can get help from the configure program at any point in time by 
emitting the following command: 
\begin{verbatim}
   ./configure --help
\end{verbatim}

Figure~\ref{fig-config} shows the options used by \walk\ from the 
\verb!configure! program. 

\centerfigbegin
\begin{tabular}{|l|l|} \hline
Option 	& Description \\ \hline
  --enable-remote       & don't try to regenerate generated files \\
  --enable-debug[=$<what>$] & enable debugging support, $<what>$ is one of ***\\
  --with-source=$<arch>$  & translate from $<arch>$ architecture,
                          one of sparc, pent \\
  --with-instrm=$<dir>$   & add instrumentation to emulator using files in $<dir>$ \\  \hline
\end{tabular}
\centerfigend{fig-config}{Configure Options}


\section{Configuring Tools from the \walk\ Framework}
At present (Aug 2001), you can generate disassemblers, interpreters 
and a PathFinder program using the \walk\ framework. 
In order to generate these tools, the framework has to be configured
using different options.  These notes describe how to \texttt{configure} 
\walk\ for different purposes.  More information about how \texttt{configure} 
works is available in Section~\ref{sec-config-process}.  
Note that in order to build a different tool you always need to 
reconfigure your system. 


\subsection{Generating Interpreters}

In order to generate interpreters, you need to first build the tool
that generates interpreters, \texttt{genemu}: 

\begin{verbatim}
   configure --with-source=sparc --enable-remote
   make dynamic/tools/genemu
\end{verbatim}
This generates the file \texttt{genemu} in the \texttt{./dynamic/tools} 
directory. 

The \texttt{genemu} tool will create an interpreter based on 
the syntax (SLED) and semantic (SSL) specifications for a machine. 
Both a C-based interpreter and a Java-based interpreter can 
be generated using \texttt{genemu}, although the Java-based interpreter
has only been tested with the SPARC specifications. 

The options available in \texttt{genemu} are: 
\begin{verbatim}
Usage: genemu [options] <sled-filename(s)> <ssl-filename>
Recognized options:
   -c  output C code [default]
   -d  disassembler only (do not generate emulator core)
   -j  output Java code
   -i  inst-filename: use inst-filename to instrument code.
   -t  test only, no code output
   -m  additionally generate a Makefile to go with the core
   -o <file>  write output to the given file
\end{verbatim}

Note that one or more SLED files can be given as input, SLED files
have the extension \texttt{.spec} and SSL files have the extension 
\texttt{.ssl}.  Also, some options have not been maintained, it is 
best to use the configure options (see next sections) or see the 
configuration files for examples of usage.  


\subsubsection{C-based Interpreters} 
To generate a C-based interpreter for a particular machine, use the 
\texttt{make} rule for that machine
(normally the name of the machine followed by ``emu''). 
For the SPARC, you would run: 
\begin{verbatim}
   make sparcemu
\end{verbatim}
This will generate \texttt{sparcemu} in the {\texttt{./dynamic/emu} 
directory, if you had configured for the SPARC machine.

To run: 
\begin{verbatim}
   cd dynamic/emu
   sparcemu ../../test/sparc/hello 
   sparcemu /bin/banner Hi
\end{verbatim}


\subsubsection{Java-based Interpreters}
To generate a Java-based interpreter for the SPARC machine, use 
the following make rule: 
\begin{verbatim}
   make dynamic/emuj
\end{verbatim}

You can then run the generated interpreter using a Java VM: 
\begin{verbatim}
   cd dynamic/emu
   java -cp . sparcmain ../../test/sparc/hello
\end{verbatim}


\subsection{Generating PathFinder}

To build a virtual machine that finds hot paths and generates 
native code for those paths while interpreting other cold 
paths, you need to configure \walk\ to generate the \texttt{pathfinder} 
virtual machine (VM)
by configuring for a particular instrumentation method to determine
hot paths within the interpreter.  For example, if you want to use   
Dynamo's next executing tail (NET) method, run the following
configure command: 

\begin{verbatim}
   configure --with-source=sparc --enable-remote \ 
             --with-instrm=dynamic/pathfinder/sparc.NET.direct/pathfinder 
\end{verbatim}

Then make the tool by building a target
with the name formed by adding a single-character prefix
to the word ``pathfinder''.
This prefix character is the first letter of the name of the machine 
on which the genberated \texttt{pathfinder} will run.  
For example, for SPARC you would run: 

\begin{verbatim}
   make spathfinder
\end{verbatim}

This generates the instrumented VM \texttt{spathfinder} in the
\texttt{./dynamic/emu} directory.
This instrumented VM uses code profiling as well as code generation
to execute code for SPARC V8.
Note, however, that the source code for the code generator
relies on some SPARC V9 instructions. 

To run: 
\begin{verbatim}
   cd dynamic/emu
   spathfinder ../../test/sparc/hello
   spathfinder ../../test/sparc/fibo-O0
   spathfinder /bin/banner Hello
\end{verbatim}

A word of caution when building interpreters and VMs.  Some of the
files used by these tools are the same and some get patched, therefore, 
it is wise to remove object files before building a new tool.  We
therefore recommend you do a \texttt{make dynclean} before you build 
your tool.  If you are getting strange errors, most likely you need 
to \texttt{make dynclean}. 
Since you need to run configure, you can do this at the same time:

\begin{verbatim}
    make dynclean
    configure --with-source=sparc ...  [whatever other options]
\end{verbatim}

You can generate other pathfinder tools for the SPARC architecture 
by using some of the other instrumentation files.  The above example 
made use of the 
\texttt{dynamic/pathfinder/sparc.NET.direct/pathfinder/profile.inst} 
instrumentation file.  
Other instrumentation files can be used, the ones in the \walk\ 
distribution are in the following locations: 

\begin{verbatim}
   dynamic/pathfinder/sparc.NET.direct/pathfinder-call/profile.inst
   dynamic/pathfinder/sparc.NET.direct/pathfinder-recursive/profile.inst
   dynamic/pathfinder/sparc.NET.direct/pathfinder.v9/profile.inst
   dynamic/pathfinder/sparc.NET.relocate/profile.inst
\end{verbatim} 


\subsection{Generating the \walk\ Debugger} 

The \walk\ debugger is a GUI debugger written in the Java language. 
The debugger was only ever tested with the SPARC architecture, 
other extensions would be needed to support the display of state 
information for other architectures.   

To configure the debugger, emit the following commands: 

\begin{verbatim}
   make dynclean
   configure --with-source=sparc --enable-dynamic --enable-remote
   make dynamic/emuDebug
\end{verbatim}
The make builds the disassembler and the interpreter for the configured 
machine, and then generates \texttt{emuDebug.class} in the 
\texttt{./dynamic/emuDebug/bin} directory.

To run the debugger, execute the bash script \texttt{emuDebug} in the 
\texttt{./dynamic/emuDebug} directory (this is a shell script file; 
make sure the first line refers to the location of your \texttt{bash} tool):  

\begin{verbatim}
   cd dynamic/emuDebug
   emuDebug ../../test/sparc/hello 
\end{verbatim}


\subsection{Building without the \texttt{--with-remote} Option}

When configuring the system \emph{without} use of the remote option 
(i.e. without \texttt{--with-remote}), the system will recreate .m and .cc 
files from the machine specifications and .c files from .y files.  

It is recommended that you do a \texttt{make realdynclean} before 
configuring without the remote option, in order to remove all generated
files that have already been stored in the distribution. 


\section{How the Configuration Process Works}
\label{sec-config-process}
A complete description of the autoconfigure process is beyond the scope of
this document; the interested reader can get more information from
publicly available
documentation such as \texttt{http://www.gnu.org/manual/autoconf/index.html}.

In brief, the developer writes a file called \texttt{configure.in}. The program
\texttt{autoconf} processes this file, and produces a script file called
\texttt{configure} that users run to configure their system. We have
already done that, so unless you need to change the configuration, you
only need to run \texttt{./configure}. If you do make a change to
\verb!configure.in!, then you should run
\begin{verbatim}
   autoconf; autoheader
\end{verbatim}

When \texttt{./configure} is run, various files are read, including a file
specific to the source machine. For example, if you configure with
\texttt{--with-source=sparc}, the file \texttt{machine/sparc/sparc.rules}
is read for SPARC-specific information. It also reads the file
\texttt{Makefile.in}. Using this information and the command line options,
\texttt{./configure} creates the file \texttt{Makefile}.
As a result, the \texttt{Makefile} isn't even
booked in. That's the main reason you need to run \texttt{./configure} as
the very first thing, before even \texttt{make}. It also means that you
should not make changes (at least, changes that are meant to be permanent)
to \verb!Makefile!; they should be made to \texttt{Makefile.in}.

Another important file created by \verb!./configure! is \verb!include/config.h!.
This file is included by \verb!include/global.h!, which in turn is included
by almost every source file. Therefore, \verb!configure! goes to some trouble
not to touch \verb!include/config.h! if there is no change to it (and it says
so at the end of the \verb!configure! run). A significant change to the
configuration (e.g., choosing a new source or target machine) will
change \verb!include/config.h!, and therefore almost everything will
have to be recompiled.

A note about the version of \texttt{autoconf}; we have found that 
version 2.9 does not work properly but version 2.13 works fine with 
our \texttt{configure} files. 


\subsection{Dependencies and \texttt{make depend}}
Building \walk requires a file called \verb!.depend!
that contains file dependencies for the system.
The first time you \verb!make! \walk,
this file won't exist
and it will be created automatically for you.
The \verb!.depend! file contains entries similar to this:

\begin{verbatim}
coverage.o: ./coverage.cc include/coverage.h include/global.h \
 include/config.h
\end{verbatim}

which says that the \verb!coverage.o! file depends on the files
\verb!./coverage.cc!, \verb!include/coverage.h!, and so on. There can be
dozens of dependencies; the above is one of the smallest. This information
takes a minute or two to generate, and so is only generated (a) by \verb!make!
itself if 
\verb!.depend! does not exist, and (b) if the user types \verb!make depend!.

It is easy to change the dependencies, e.g. by adding a
\verb!#include! line to a source file. If you do this, and forget to run
\verb!make depend!, then you can end up with very subtle make problems that
are very hard to track down. For example, suppose you add
``\verb!#include "foo.h"!'' to the \verb!worker.cc! source file,
so that \verb!worker.cc! can use the last virtual method in class \verb!foo!.
Everything compiles and works fine. A week later, you add a virtual method to
the middle of \verb!class foo!. The \verb!.depend! file doesn't have the
dependency for \verb!worker.cc! on \verb!foo.h!, and so \verb!worker.o! isn't
remade. The code in \verb!worker.o! now calls the second last method
in \verb!class foo!, instead of the correct final method! However,
you are not thinking about \verb!worker.cc! now, since your latest changes
are elsewhere. This sort of problem can take a long time to diagnose.

One solution is to ``\verb!make clean!'' as soon as you get unexpected
results. However, you can save a lot of time if instead you just
\verb!make depend; make! instead. In fact, it's a good idea to run
\verb!make depend! regularly, or after any significant change to the source
files.

\subsection{Warnings from \texttt{make}}

During the making of \walk, it is normal to see quite a lot of output. We try to
ensure that ordinary warnings from gcc are prevented, but some warnings are much
harder to suppress, and some warnings are quite normal. For example:

\verb!typeAnalysis/typeAnalysis.y contains 2 shift/reduce conflicts.!

These are normal, and the bison++ parser automatically resolves these conflicts
in a sensible way.


\subsection{Where the Makefile Rules Are} 
The Makefile is composed of make rules
that come from a variety of different sources:

\begin{itemize}
\item The core rules are in the top level \texttt{Makefile.in} file, 
\item Machine-specific rules are in the respective machine directory 
	with the extension \texttt{.rules};
      e.g., \texttt{machine/sparc/sparc.rules},
\item Instrumentation rules are in a subdirectory \texttt{pathfinder/make.rules}
	under the respective instrumentation directory; e.g.,
	\texttt{pathfinder/sparc.NET.direct/pathfinder/make.rules}. 
\end{itemize}


\section{The \walk\ Regression Test Suite}
The \walk\ framework includes a set of regression tests for
generated interpreters.
Tests have been written for the SPARC emulator only.

The script that is used for testing an interpreter is called 
\texttt{dynamic/test/interp-regression.tcl}.
This is a Tcl script that allows new tests to be added easily.
Existing tests can also be modified easily.
Each test includes the test's name and expected result.
The script runs each test and compares its output against that expected.
If these are not the same, it reports the failure and gives the actual result.
The default is to run all tests,
but you can specify which tests to run,
or which to skip, by giving a regular expression pattern
that is matched against test names.
At the end of all the tests, there is a report
on the number of tests run, and how many passed, failed, or were skipped.

The tests each run one of the SPEC95 benchmark programs.
Up to four regression tests can be run for each SPEC95 program.
Both optimized (-O4) and unoptimized (-O) versions of each program are run.
It is also possible to run each version of the program
using both the SPEC95 ``test'' and ``reference'' data sets.
By default, only the ``test'' data set is used
since it requires less time.
The other tests that use the ``reference'' data set
are marked as having the ``refInput'' contraint;
they are only run if you specify \texttt{-constraints refInput}
on the command line that runs the regression tests (see below).

The expected output of each test
is the total number of instructions executed.
This count is sensitive to the specific versions of the shared libraries that
are used.
As a result, you can expect the tests to fail if they are run on
a machine other than that used to get the original expected instruction count.
We used a Sun 420R with 4 CPUs and 4GB of memory, running Solaris 8. 


\subsection{Running the Regression Tests} 
Tests can be run sequentially (one after another) or in parallel.
This section discusses how to run the tests sequentially.
Section~\ref{sec-mt-tests} describes how to
run the tests concurrently.

To run all SPARC interpreter tests using the ``test'' data sets (the default):
\begin{verbatim}
    $ cd <workspace>
    $ cd dynamic/test
    $ tclsh interp-regression.test sparc
\end{verbatim}

To run only the SPARC interpreter tests that run the SPEC95 ``go'' program,
you can specify a pattern on the command line.
Only tests (``test'' data sets only)
with names that match the pattern are run:
\begin{verbatim}
    cd dynamic/test
    tclsh interp-regression.test sparc -match 'go*'
\end{verbatim}

To skip all SPARC interpreter tests whose names match a pattern
(again ``test'' data sets only), run:
\begin{verbatim}
    cd dynamic/test
    tclsh interp-regression.test sparc -skip 'go* ijpeg*'
\end{verbatim}

To run all SPARC interpreter tests
including those that use the ``reference'' data sets:
\begin{verbatim}
    cd dynamic/test
    tclsh interp-regression.test sparc -constraints refInput
\end{verbatim}


\subsection{Running the Tests in Parallel} 
\label{sec-mt-tests}

To run the tests in parallel,
the two scripts \texttt{dynamic/test/mt-tests.tcl}
and \texttt{dynamic/test/summarize-mt-tests.tcl}
are used.
\texttt{mt-tests.tcl} takes the same command line arguments
as \texttt{interp-regression.tcl}
and forks processes to do the various tests in parallel.
It creates a directory under
\texttt{dynamic/test} with a name like \texttt{walktest-Oct-24-2001-19:15:06}
that contains an ``*.out'' file holding the output
from each of the forked tests.
One such file, for example, might be \texttt{go.v8.base.test.out}.

The second script \texttt{summarize-mt-tests.tcl}
creates a file that summarizes the results of all the tests.
It takes one command line argument,
the name of the directory created by \texttt{mt-tests.tcl},
and creates a file \texttt{summary.txt} in that
directory with the concatenated results of the various tests.

For example, to run all SPARC interpreter tests except for ``go''
using the default ``test'' data sets, do the following.

\begin{verbatim}
    $ cd <workspace>/dynamic/test
    $ tclsh mt-tests.tcl sparc -skip 'go*'
    Parallel Walkabout tests
    Writing results into directory walktest-Oct-24-2001-16:10:44
    Forked process 7952 to execute test compress.v8.peak.test
    Forked process 7953 to execute test compress.v8.base.test
    Forked process 7954 to execute test perl.v8.peak.test
    Forked process 7955 to execute test perl.v8.base.test
    Forked process 7956 to execute test go.v8.peak.test
    Forked process 7957 to execute test go.v8.base.test
    Forked process 7958 to execute test ijpeg.v8.peak.test
    Forked process 7964 to execute test ijpeg.v8.base.test
    $ tclsh summarize-mt-tests.tcl walktest-Oct-24-2001-16:10:44
    Summarizing parallel Walkabout test results into walktest-Oct-24-2001-16:10:44/summary.txt
    appending file compress.v8.base.test.out
    ...
    $ vi walktest-Oct-24-2001-16:10:44/summary.txt
\end{verbatim}

  	% configuration notes (how to compile walkabout)

}{
% This is the "else" set of brackets

	%
% 29 Aug 01 - Cristina: created 
% 24 Oct 01 - Brian: Added testing section. Made various edits. 
%  8 Jan 01 - Cristina: updated
%

\chapter{Building Walkabout}
\label{ch-config}

{\small
\begin{flushright}
Documentation: Cristina [Aug 01, Jan 02]
\end{flushright}
}

This chapter contains notes on how to configure the \walk\ framework
for a given platform.  
The tools that can be built within the \walk\ framework include 
an emulator, a disassembler, a pathfinder and a debugger. 
Most of the examples are for the SPARC architecture as this was 
our development platform.  


\section{Compilers and Tools Needed to build \walk}
We use gcc 2.95.3, however, we do not make use of any of the new 
classes that are not available in 2.95-2, such as sstream. 
Note that we make use of namespaces sparsely in the code and these
are not supported by the gcc 2.8.1 version of the compiler, 
but they are in egcs-1.1.2 (gcc 2.91.66). 

For debugging, gdb 5.0 works well with gcc 2.95.3.


\subsection{Special tools needed to build \walk}

\walk\ has many source files that are generated from other source files,
or from specifications. It is possible to make \walk\ without installing
these tools, but if you want to make significant changes to \walk, you will
need those tools.

To make \walk\ without the special tools, use the \texttt{--enable-remote}
configuration script (see above).

The special tools are as follows.

\begin{itemize}
\item The New Jersey Machine Code Toolkit, ML version. This tool reads machine
specifications, and in association with a matcher (\texttt{.m}) file, generates
binary decoders. For details and downloading, see
\texttt{http://www.eecs.harvard.edu/~nr/toolkit/ml.html}.
\item Bison++ and Flex++, C++ versions. Note that the GNU tool bison++ is
{\it not} suitable; \walk\ needs the special versions from France, which are
C++ aware. If you get lots of errors from running bison++, you have probably
got the wrong version! Download these tools from
\texttt{ ftp://ftp.th-darmstadt.de/pub/programming/languages/C++/tools/flex++bison++/LATEST/}
You might also try downloading these tools from one of the various mirror sites such as
\texttt{http://sunsite.bilkent.edu.tr/pub/languages/c++/tools/flex++bison++/LATEST/}.
To test if you have the correct version, you should get results similar to:
\begin{verbatim}
%  bison++ --version
bison++ Version 1.21-7, adapted from GNU bison by coetmeur@icdc.fr
\end{verbatim}
If searching the web for these tooks, include the author's name ("coetmeur")
as a keyword.
\item The Tcl shell (\texttt{tclsh}). This tool is only needed to run the
regression test script (\texttt{test/regression.test}).
\texttt{tclsh} and the \texttt{tcltest} package are part of
Tcl/Tk releases 8.0 and newer.
You may well find that these are already installed on
your Linux or other system. Otherwise, see web pages such as
\texttt{http://www.sco.com/Technology/tcl/Tcl.html}.
\end{itemize}


\section{Configuration Notes}
In order to instantiate a translator out of the \walk\ framework, 
you need to configure \walk\ to run on your host machine by instantiating 
a set of source and target machines.  Figure~\ref{fig-mach-names} 
lists the names used within \walk\ to describe machine specifications, and 
the associated instruction set version that is specified.  

\centerfigbegin
\begin{tabular}{|l|l|} \hline
Name	& Description \\ \hline
sparc	& SPARC V8 (integers and floats) \\
pent	& 80386 (integers and floats) \\ \hline
\end{tabular}
\centerfigend{fig-mach-names}{Names of Machines and Versions Supported 
	by the \walk\ Framework}

You can get help from the configure program at any point in time by 
emitting the following command: 
\begin{verbatim}
   ./configure --help
\end{verbatim}

Figure~\ref{fig-config} shows the options used by \walk\ from the 
\verb!configure! program. 

\centerfigbegin
\begin{tabular}{|l|l|} \hline
Option 	& Description \\ \hline
  --enable-remote       & don't try to regenerate generated files \\
  --enable-debug[=$<what>$] & enable debugging support, $<what>$ is one of ***\\
  --with-source=$<arch>$  & translate from $<arch>$ architecture,
                          one of sparc, pent \\
  --with-instrm=$<dir>$   & add instrumentation to emulator using files in $<dir>$ \\  \hline
\end{tabular}
\centerfigend{fig-config}{Configure Options}


\section{Configuring Tools from the \walk\ Framework}
At present (Aug 2001), you can generate disassemblers, interpreters 
and a PathFinder program using the \walk\ framework. 
In order to generate these tools, the framework has to be configured
using different options.  These notes describe how to \texttt{configure} 
\walk\ for different purposes.  More information about how \texttt{configure} 
works is available in Section~\ref{sec-config-process}.  
Note that in order to build a different tool you always need to 
reconfigure your system. 


\subsection{Generating Interpreters}

In order to generate interpreters, you need to first build the tool
that generates interpreters, \texttt{genemu}: 

\begin{verbatim}
   configure --with-source=sparc --enable-remote
   make dynamic/tools/genemu
\end{verbatim}
This generates the file \texttt{genemu} in the \texttt{./dynamic/tools} 
directory. 

The \texttt{genemu} tool will create an interpreter based on 
the syntax (SLED) and semantic (SSL) specifications for a machine. 
Both a C-based interpreter and a Java-based interpreter can 
be generated using \texttt{genemu}, although the Java-based interpreter
has only been tested with the SPARC specifications. 

The options available in \texttt{genemu} are: 
\begin{verbatim}
Usage: genemu [options] <sled-filename(s)> <ssl-filename>
Recognized options:
   -c  output C code [default]
   -d  disassembler only (do not generate emulator core)
   -j  output Java code
   -i  inst-filename: use inst-filename to instrument code.
   -t  test only, no code output
   -m  additionally generate a Makefile to go with the core
   -o <file>  write output to the given file
\end{verbatim}

Note that one or more SLED files can be given as input, SLED files
have the extension \texttt{.spec} and SSL files have the extension 
\texttt{.ssl}.  Also, some options have not been maintained, it is 
best to use the configure options (see next sections) or see the 
configuration files for examples of usage.  


\subsubsection{C-based Interpreters} 
To generate a C-based interpreter for a particular machine, use the 
\texttt{make} rule for that machine
(normally the name of the machine followed by ``emu''). 
For the SPARC, you would run: 
\begin{verbatim}
   make sparcemu
\end{verbatim}
This will generate \texttt{sparcemu} in the {\texttt{./dynamic/emu} 
directory, if you had configured for the SPARC machine.

To run: 
\begin{verbatim}
   cd dynamic/emu
   sparcemu ../../test/sparc/hello 
   sparcemu /bin/banner Hi
\end{verbatim}


\subsubsection{Java-based Interpreters}
To generate a Java-based interpreter for the SPARC machine, use 
the following make rule: 
\begin{verbatim}
   make dynamic/emuj
\end{verbatim}

You can then run the generated interpreter using a Java VM: 
\begin{verbatim}
   cd dynamic/emu
   java -cp . sparcmain ../../test/sparc/hello
\end{verbatim}


\subsection{Generating PathFinder}

To build a virtual machine that finds hot paths and generates 
native code for those paths while interpreting other cold 
paths, you need to configure \walk\ to generate the \texttt{pathfinder} 
virtual machine (VM)
by configuring for a particular instrumentation method to determine
hot paths within the interpreter.  For example, if you want to use   
Dynamo's next executing tail (NET) method, run the following
configure command: 

\begin{verbatim}
   configure --with-source=sparc --enable-remote \ 
             --with-instrm=dynamic/pathfinder/sparc.NET.direct/pathfinder 
\end{verbatim}

Then make the tool by building a target
with the name formed by adding a single-character prefix
to the word ``pathfinder''.
This prefix character is the first letter of the name of the machine 
on which the genberated \texttt{pathfinder} will run.  
For example, for SPARC you would run: 

\begin{verbatim}
   make spathfinder
\end{verbatim}

This generates the instrumented VM \texttt{spathfinder} in the
\texttt{./dynamic/emu} directory.
This instrumented VM uses code profiling as well as code generation
to execute code for SPARC V8.
Note, however, that the source code for the code generator
relies on some SPARC V9 instructions. 

To run: 
\begin{verbatim}
   cd dynamic/emu
   spathfinder ../../test/sparc/hello
   spathfinder ../../test/sparc/fibo-O0
   spathfinder /bin/banner Hello
\end{verbatim}

A word of caution when building interpreters and VMs.  Some of the
files used by these tools are the same and some get patched, therefore, 
it is wise to remove object files before building a new tool.  We
therefore recommend you do a \texttt{make dynclean} before you build 
your tool.  If you are getting strange errors, most likely you need 
to \texttt{make dynclean}. 
Since you need to run configure, you can do this at the same time:

\begin{verbatim}
    make dynclean
    configure --with-source=sparc ...  [whatever other options]
\end{verbatim}

You can generate other pathfinder tools for the SPARC architecture 
by using some of the other instrumentation files.  The above example 
made use of the 
\texttt{dynamic/pathfinder/sparc.NET.direct/pathfinder/profile.inst} 
instrumentation file.  
Other instrumentation files can be used, the ones in the \walk\ 
distribution are in the following locations: 

\begin{verbatim}
   dynamic/pathfinder/sparc.NET.direct/pathfinder-call/profile.inst
   dynamic/pathfinder/sparc.NET.direct/pathfinder-recursive/profile.inst
   dynamic/pathfinder/sparc.NET.direct/pathfinder.v9/profile.inst
   dynamic/pathfinder/sparc.NET.relocate/profile.inst
\end{verbatim} 


\subsection{Generating the \walk\ Debugger} 

The \walk\ debugger is a GUI debugger written in the Java language. 
The debugger was only ever tested with the SPARC architecture, 
other extensions would be needed to support the display of state 
information for other architectures.   

To configure the debugger, emit the following commands: 

\begin{verbatim}
   make dynclean
   configure --with-source=sparc --enable-dynamic --enable-remote
   make dynamic/emuDebug
\end{verbatim}
The make builds the disassembler and the interpreter for the configured 
machine, and then generates \texttt{emuDebug.class} in the 
\texttt{./dynamic/emuDebug/bin} directory.

To run the debugger, execute the bash script \texttt{emuDebug} in the 
\texttt{./dynamic/emuDebug} directory (this is a shell script file; 
make sure the first line refers to the location of your \texttt{bash} tool):  

\begin{verbatim}
   cd dynamic/emuDebug
   emuDebug ../../test/sparc/hello 
\end{verbatim}


\subsection{Building without the \texttt{--with-remote} Option}

When configuring the system \emph{without} use of the remote option 
(i.e. without \texttt{--with-remote}), the system will recreate .m and .cc 
files from the machine specifications and .c files from .y files.  

It is recommended that you do a \texttt{make realdynclean} before 
configuring without the remote option, in order to remove all generated
files that have already been stored in the distribution. 


\section{How the Configuration Process Works}
\label{sec-config-process}
A complete description of the autoconfigure process is beyond the scope of
this document; the interested reader can get more information from
publicly available
documentation such as \texttt{http://www.gnu.org/manual/autoconf/index.html}.

In brief, the developer writes a file called \texttt{configure.in}. The program
\texttt{autoconf} processes this file, and produces a script file called
\texttt{configure} that users run to configure their system. We have
already done that, so unless you need to change the configuration, you
only need to run \texttt{./configure}. If you do make a change to
\verb!configure.in!, then you should run
\begin{verbatim}
   autoconf; autoheader
\end{verbatim}

When \texttt{./configure} is run, various files are read, including a file
specific to the source machine. For example, if you configure with
\texttt{--with-source=sparc}, the file \texttt{machine/sparc/sparc.rules}
is read for SPARC-specific information. It also reads the file
\texttt{Makefile.in}. Using this information and the command line options,
\texttt{./configure} creates the file \texttt{Makefile}.
As a result, the \texttt{Makefile} isn't even
booked in. That's the main reason you need to run \texttt{./configure} as
the very first thing, before even \texttt{make}. It also means that you
should not make changes (at least, changes that are meant to be permanent)
to \verb!Makefile!; they should be made to \texttt{Makefile.in}.

Another important file created by \verb!./configure! is \verb!include/config.h!.
This file is included by \verb!include/global.h!, which in turn is included
by almost every source file. Therefore, \verb!configure! goes to some trouble
not to touch \verb!include/config.h! if there is no change to it (and it says
so at the end of the \verb!configure! run). A significant change to the
configuration (e.g., choosing a new source or target machine) will
change \verb!include/config.h!, and therefore almost everything will
have to be recompiled.

A note about the version of \texttt{autoconf}; we have found that 
version 2.9 does not work properly but version 2.13 works fine with 
our \texttt{configure} files. 


\subsection{Dependencies and \texttt{make depend}}
Building \walk requires a file called \verb!.depend!
that contains file dependencies for the system.
The first time you \verb!make! \walk,
this file won't exist
and it will be created automatically for you.
The \verb!.depend! file contains entries similar to this:

\begin{verbatim}
coverage.o: ./coverage.cc include/coverage.h include/global.h \
 include/config.h
\end{verbatim}

which says that the \verb!coverage.o! file depends on the files
\verb!./coverage.cc!, \verb!include/coverage.h!, and so on. There can be
dozens of dependencies; the above is one of the smallest. This information
takes a minute or two to generate, and so is only generated (a) by \verb!make!
itself if 
\verb!.depend! does not exist, and (b) if the user types \verb!make depend!.

It is easy to change the dependencies, e.g. by adding a
\verb!#include! line to a source file. If you do this, and forget to run
\verb!make depend!, then you can end up with very subtle make problems that
are very hard to track down. For example, suppose you add
``\verb!#include "foo.h"!'' to the \verb!worker.cc! source file,
so that \verb!worker.cc! can use the last virtual method in class \verb!foo!.
Everything compiles and works fine. A week later, you add a virtual method to
the middle of \verb!class foo!. The \verb!.depend! file doesn't have the
dependency for \verb!worker.cc! on \verb!foo.h!, and so \verb!worker.o! isn't
remade. The code in \verb!worker.o! now calls the second last method
in \verb!class foo!, instead of the correct final method! However,
you are not thinking about \verb!worker.cc! now, since your latest changes
are elsewhere. This sort of problem can take a long time to diagnose.

One solution is to ``\verb!make clean!'' as soon as you get unexpected
results. However, you can save a lot of time if instead you just
\verb!make depend; make! instead. In fact, it's a good idea to run
\verb!make depend! regularly, or after any significant change to the source
files.

\subsection{Warnings from \texttt{make}}

During the making of \walk, it is normal to see quite a lot of output. We try to
ensure that ordinary warnings from gcc are prevented, but some warnings are much
harder to suppress, and some warnings are quite normal. For example:

\verb!typeAnalysis/typeAnalysis.y contains 2 shift/reduce conflicts.!

These are normal, and the bison++ parser automatically resolves these conflicts
in a sensible way.


\subsection{Where the Makefile Rules Are} 
The Makefile is composed of make rules
that come from a variety of different sources:

\begin{itemize}
\item The core rules are in the top level \texttt{Makefile.in} file, 
\item Machine-specific rules are in the respective machine directory 
	with the extension \texttt{.rules};
      e.g., \texttt{machine/sparc/sparc.rules},
\item Instrumentation rules are in a subdirectory \texttt{pathfinder/make.rules}
	under the respective instrumentation directory; e.g.,
	\texttt{pathfinder/sparc.NET.direct/pathfinder/make.rules}. 
\end{itemize}


\section{The \walk\ Regression Test Suite}
The \walk\ framework includes a set of regression tests for
generated interpreters.
Tests have been written for the SPARC emulator only.

The script that is used for testing an interpreter is called 
\texttt{dynamic/test/interp-regression.tcl}.
This is a Tcl script that allows new tests to be added easily.
Existing tests can also be modified easily.
Each test includes the test's name and expected result.
The script runs each test and compares its output against that expected.
If these are not the same, it reports the failure and gives the actual result.
The default is to run all tests,
but you can specify which tests to run,
or which to skip, by giving a regular expression pattern
that is matched against test names.
At the end of all the tests, there is a report
on the number of tests run, and how many passed, failed, or were skipped.

The tests each run one of the SPEC95 benchmark programs.
Up to four regression tests can be run for each SPEC95 program.
Both optimized (-O4) and unoptimized (-O) versions of each program are run.
It is also possible to run each version of the program
using both the SPEC95 ``test'' and ``reference'' data sets.
By default, only the ``test'' data set is used
since it requires less time.
The other tests that use the ``reference'' data set
are marked as having the ``refInput'' contraint;
they are only run if you specify \texttt{-constraints refInput}
on the command line that runs the regression tests (see below).

The expected output of each test
is the total number of instructions executed.
This count is sensitive to the specific versions of the shared libraries that
are used.
As a result, you can expect the tests to fail if they are run on
a machine other than that used to get the original expected instruction count.
We used a Sun 420R with 4 CPUs and 4GB of memory, running Solaris 8. 


\subsection{Running the Regression Tests} 
Tests can be run sequentially (one after another) or in parallel.
This section discusses how to run the tests sequentially.
Section~\ref{sec-mt-tests} describes how to
run the tests concurrently.

To run all SPARC interpreter tests using the ``test'' data sets (the default):
\begin{verbatim}
    $ cd <workspace>
    $ cd dynamic/test
    $ tclsh interp-regression.test sparc
\end{verbatim}

To run only the SPARC interpreter tests that run the SPEC95 ``go'' program,
you can specify a pattern on the command line.
Only tests (``test'' data sets only)
with names that match the pattern are run:
\begin{verbatim}
    cd dynamic/test
    tclsh interp-regression.test sparc -match 'go*'
\end{verbatim}

To skip all SPARC interpreter tests whose names match a pattern
(again ``test'' data sets only), run:
\begin{verbatim}
    cd dynamic/test
    tclsh interp-regression.test sparc -skip 'go* ijpeg*'
\end{verbatim}

To run all SPARC interpreter tests
including those that use the ``reference'' data sets:
\begin{verbatim}
    cd dynamic/test
    tclsh interp-regression.test sparc -constraints refInput
\end{verbatim}


\subsection{Running the Tests in Parallel} 
\label{sec-mt-tests}

To run the tests in parallel,
the two scripts \texttt{dynamic/test/mt-tests.tcl}
and \texttt{dynamic/test/summarize-mt-tests.tcl}
are used.
\texttt{mt-tests.tcl} takes the same command line arguments
as \texttt{interp-regression.tcl}
and forks processes to do the various tests in parallel.
It creates a directory under
\texttt{dynamic/test} with a name like \texttt{walktest-Oct-24-2001-19:15:06}
that contains an ``*.out'' file holding the output
from each of the forked tests.
One such file, for example, might be \texttt{go.v8.base.test.out}.

The second script \texttt{summarize-mt-tests.tcl}
creates a file that summarizes the results of all the tests.
It takes one command line argument,
the name of the directory created by \texttt{mt-tests.tcl},
and creates a file \texttt{summary.txt} in that
directory with the concatenated results of the various tests.

For example, to run all SPARC interpreter tests except for ``go''
using the default ``test'' data sets, do the following.

\begin{verbatim}
    $ cd <workspace>/dynamic/test
    $ tclsh mt-tests.tcl sparc -skip 'go*'
    Parallel Walkabout tests
    Writing results into directory walktest-Oct-24-2001-16:10:44
    Forked process 7952 to execute test compress.v8.peak.test
    Forked process 7953 to execute test compress.v8.base.test
    Forked process 7954 to execute test perl.v8.peak.test
    Forked process 7955 to execute test perl.v8.base.test
    Forked process 7956 to execute test go.v8.peak.test
    Forked process 7957 to execute test go.v8.base.test
    Forked process 7958 to execute test ijpeg.v8.peak.test
    Forked process 7964 to execute test ijpeg.v8.base.test
    $ tclsh summarize-mt-tests.tcl walktest-Oct-24-2001-16:10:44
    Summarizing parallel Walkabout test results into walktest-Oct-24-2001-16:10:44/summary.txt
    appending file compress.v8.base.test.out
    ...
    $ vi walktest-Oct-24-2001-16:10:44/summary.txt
\end{verbatim}


}


\bibliography{/home/cristina/tex/biblio/cristina,/home/cristina/tex/biblio/translation}
\bibliographystyle{plain}
\addcontentsline{toc}{chapter}{Bibliography}


\end{document}

