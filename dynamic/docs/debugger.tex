\chapter{Debugger}
\label{ch-debugger}

{\small
\begin{flushright}
Design: Bernard Wong [Oct 2001]; Documentation: Bernard Wong [Jan 2002]
\end{flushright}
}

The experiments with the emulator have so far been very positive and show much 
promise in enabling a more robust dynamic method to perform binary translation. 
Currently, the emulator works very well with most small conventional user 
programs on the SPARC. However, large amounts of time was required to debug the 
emulator in order for it to reach this working state for the SPARC Solaris 
platform. This is because it is very difficult to debug the emulator, as it 
often requires the programmer to read pages and pages of SPARC assembly code 
to spot the one mistake the emulator makes. 

The goal of the graphical debugger for the emulator is to reduce the time and 
effort necessary in debugging the emulator for each new platform which it is to 
support. This goal drives the two main necessary requirements in the design of 
the debugger. These requirements are the following:

\begin{enumerate}
\item To allow easier and quicker debugging of the emulator, and  
\item To run and support every platform which the emulator supports.
\end{enumerate}

Please note that the emulator debugger currently depends on the the Unsafe 
package which is only available with Java 1.4 (in Beta testing at the 
time of writing).


\section{Overview of Design}

The debugger is written in the Java language and makes use of the Swing 
package. The program can be broken up into the following sections:

\begin{enumerate}
\item Graphical Section - \texttt{emuDebug.java}
\item Emulator Connection Section - \texttt{emuProcess.java}, 
		\texttt{emuLib.skel}
\item Disassembler Section - \texttt{disasm.java}
\item Debugger Preprocessor - \texttt{emuDebugGen} 
\end{enumerate}


\subsection{Graphical Section}

The graphical section, as its name implies, is a collection of classes that 
deals with the different graphical parts of the debugger. Any modification to 
GUI should be made to this section. There are currently 5 main windows to the 
debugger: Disasm Output, Command Window, Register Window, Trace Window and Misc.
Window. 

Some of the GUI components' functionality are not currently implemented. These 
include the Relocation radio buttons, the Trace Window, the Float Registers 
frame and the View at Mem Address box. The Relocation buttons and Trace Window 
are intended for use by the PathFinder, as it would give the PathFinder tool the
ability to display addresses and assembly instructions of the collected traces 
which would greatly help debugging that tool as well.


\subsection{Emulator Connection Section}

The emulator connection section is a separate Java program which interacts with 
the emulator and communicates with the rest of the debugger via sockets. Every 
time a breakpoint is reached the emulator connection section will send the 
current emulator state information to the graphical section serving as a bridge
between the debugger and the emulator. 

The main reason for separating the emulator connection section into a separate 
program from the graphical section is because of the stability of the emulator.
During many of the development stages of the emulator, it would often crash if 
it encountered certain combination of instructions. If the emulator, which is a 
native C program, crashes, it will cause the emulator connection section to also
crash as the emulator connection section is connected to the emulator via 
the Java Native Interface (JNI). 
By separating the emulator connection section from the graphical section, it 
allows the debugger to gracefully recover from the crash and collect all the 
relevant emulator state information just before the crash occurred without 
requiring a restart of the entire debugger. 


\subsection{Disassembler Section}

The disassembler section uses the existing automatically generated disassembler 
and formats the disassembled information into a format compatible with the 
Graphical Section. Since the automatically generated disassembler is generated 
by the same tool as the automatically generated emulator, any platform which the
emulator supports would also be supported by the disassembler. Therefore, using
this disassembler will help the debugger satisfy its requirement to support all
the platforms which the emulator supports.


\subsection{Debugger Preprocessor}

Finally, the debugger preprocessor is an effort to allow the debugger to easily 
support every platform which the emulator supports. Places in the debugger that 
require platform specific information are marked with special preprocessor 
symbols that are later replaced with platform specific code that is  
automatically generated via the use of specification files. A very simple parser
used to perform the replacements can be found in the \texttt{emuDebugGen} 
directory. The specification file is \texttt{machine/sparc/emuDebugSPARC.spec} 
(for SPARC machines) and contains the following section headings: 
\texttt{ConditionCodes}, \texttt{IntegerRegisters}, \texttt{ProgramCounter}, 
and \texttt{MiscRegisters}.  The specification definition is currently 
only sufficient to support SPARC instructions. Additional headings support will 
need to be added in order to support other platforms.

Performing the command

\begin{verbatim}
java CodeGen src/machine/sparc/emuDebugSPARC.spec \
     src/machine/sparc/emuDebugSPARC.m \
     src/machine/sparc/emuLib.skel
\end{verbatim}

will generate the final platform specific Java files from the skeleton and 
specification files. The included make file will already perform the necessary 
Java files generation.

\section{Current Status}
The current version of the debugger allows for the execution of the emulator to 
be controlled via the graphical panels of the debugger. The emulator can be told
to emulate one instruction at a time (stepping through emulated machine 
instructions), or can be told to stop at breakpoints specified by memory 
addresses. Breakpoints can be added graphically as the assembly instructions of 
the executing program are shown. However, the assembly instructions when the 
program enters a library file is not shown. Therefore it is best to compile the 
executable to be emulated statically during the debugging phases. At each 
breakpoint, the current integer register, control flags, PC and nPC and other 
important register values are shown. The debugger can also restart the emulator
at any point of execution.

Currently, only the SPARC emulator is supported by the debugger. In order to 
support other platforms, the specification file definition will need to be 
expanded. A more mature parser may need to be written in order to easily parse 
an expanded specification definition (the current parser is only meant to 
quickly test the feasibility of using specification files to generate Java 
files). 

Support for PathFinder is also not currently implemented. Supporting 
Pathfinder with the debugger would be a very worthwhile feature as it would 
greatly help to debug the complex PathFinder code.


