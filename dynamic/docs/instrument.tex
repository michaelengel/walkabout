\chapter{Instrumentation of an Interpreter via Specifications}
\label{ch-genemu-instr}

{\small
\begin{flushright}
Design: Cristina, David; Implementation: David; 
Documentation: David Ung [May 2001], Cristina [Jan 2002]
\end{flushright}
}

This chapter describes the specification file format used to automatically 
add instrumentation code to the \walk\ based emulator (described in 
Chapter~\ref{ch-genemu}). 
Example instrumentation files are also given to demonstrate the type of 
instrumentation that it can create and how it is integrated into the emulator.


\section{Instrumentation}
The goal of this research is to determine an inexpensive and easy to use way 
to add instrumentation to the emulator.  The type of instrumentation to be 
added describes ways to identify what sections of the source program are hot 
(i.e. frequently executed), so that code generation can be done to improve 
the overall performance of the execution.

Existing tools such as EEL~\cite{Laru95} and ATOM~\cite{Eust95} provide an 
interface to add code by specifying the level where instrumentation should 
take place and what to instrument.  
Such tools reconstructs the application to be instrumented 
to an intermediate representation in memory and then modifies its structure 
(through CFGs and basic blocks) to add the instrumentation into the application.
Finally, the binary is rebuilt and an instrumented version is emitted.

There are two fundamental problems with the approach described above when 
trying to add instrumentation to the \walk\ emulator:

\begin{enumerate}
\item When the emulator is running, it is emulating the behaviour of a program. 
The execution of the program determines which paths to take during runtime and 
hence indirectly affects which part of the emulator is invoked.  Adding 
instrumentation to the emulator will instrument the emulator.  
Although this can indirectly give information about the runtime behaviour of 
the source program, the information gathered will be more in the scope of the 
emulator, thus losing emphasis on the source program.  
In particular to the attempt to find hot traces, instrumentating the emulator 
may tell us that the function \texttt{executeBNE()} is hot, but the 
information about which instruction cause this in the source program is 
not revealed.  
It is possible to obtain information about the source program through the 
instrumenting the emulator, but the task is not an easy one.  It requires 
knowledge about the internal workings of the emulator and makes the works of 
instrumentation difficult to use.

\item The emulator is a very low level data processor.  Instead of multiple 
level of abstractions found in existing tools, the only abstraction of the 
emulator is at the instruction level.  
This low level abstraction greatly limits the amount of calls that can be 
made to predefined functions provided by high level instrumentation tools.  
For example, high level functions such as FOREACH\_EDGE() and FOREACH\_BB().
\end{enumerate}

To provide a flexible and powerful instrumentation at the instruction level, 
the emulator itself should provide the instrumentation.  Since the emulator 
is automatically generated, this motivates the idea of automatically adding 
instrumentation code as part of the emulator. 
Being part of the emulator code, the instrumentation has access to variable 
and locations internal to the emulator.  This approach allows direct control 
over what to instrument.  The goal is to instrument the source program, not the 
emulator. 


\subsection{Existing instrumentation tools}
The use of instrumentation provides opportunities in binary editing, emulation,
observation, program comprehension and optimization.  Although many tools exist
that can modify binaries, the implementation of the actual code modification 
are typically fused in detail with the application or executable itself.  
But several tools exist that provide a high level interface (typically 
through a set of libraries) to easily access its instrumentation facilities.  
The following are some examples of such tools:

\begin{enumerate}
\item Srivastava and Wall's OM system~\cite{Dec94}, a library for binary 
	modification.  It requires relocation from object files to analyse 
	control structure and to relocate edited code.
\item ATOM~\cite{Eust95} provide an interface to the OM system.  
	Very high level of abstraction, simplifies the writing of tools.
\item QPT~\cite{Laru94} by Larus and Ball, a profiling and tracing tool.
\item EEL~\cite{Laru95}, also a library for building tools to analyze and 
	modify binaries.
\end{enumerate}

Both EEL and ATOM are large libraries that provide a rich set of routines for 
instrumentation. 
Different levels of the abstraction in these tools allows control over what 
level the tool wants to instrument through calls to those library rountines.  
The following example EEL code shows the use of the library:

\begin{verbatim}
executable* exec = new executable("test_program");
exec->read_contents();

rountine* r;
FOREACH_ROUTINE (r, exec->rountines()) {
  cfg* g = r->get_control_graph();
  bb* b;
  FOREACH_BB(b, g->blocks()) {
    edge* e;
    FOREACH_edge(e, b->succ()) {
      count_branch(e);
    } 
  }
}
\end{verbatim}

The types \texttt{executable}, \texttt{rountine}, \texttt{cfg}, \texttt{bb} and 
\texttt{edge} are data structures provided by the library for different level of
abstraction of the binary.  The tools simply make calls to library routines 
such as \texttt{read\_contents()}, \texttt{FOREACH\_BB} and 
\texttt{get\_control\_graph()}.  The only function that 
needs to be written by the tool builder is \texttt{count\_branch()}.
The concept in ATOM is similar to EEL in that the tool builder make uses of 
library rountine to access the different levels of abstraction in the binary.


\section{Instrumentation specification}

The instrumentation code is written in a separate specification file that is 
linked as part of the emulator at the time of generating the New Jersey 
Machine Code (NJMC) matching file.
Code can be added at the instruction level under the \texttt{DEFINITION} 
section.  
The list of instructions that you wish to act on is specified as a table.  
Instrumentation code is then specified for a table by adding relevant code with
respect to the main body of their emulation rountines.
For example, if you want to count the number of times a particular set of 
branches is taken in an X86 program, the instructions to be monitored are 
specified in a table as follows:

\begin{verbatim}
DEFINITION

jump32s	[ "JVA", "JVNBE", "JVAE", "JVNB", "JVB", "JVNAE", "JVBE",
          "JVNA", "JVC", "JVCXZ", "JVE", "JVZ", "JVG", "JVNLE", "JVGE",
          "JVNL", "JVL", "JVNGE", "JVLE", "JVNG", "JVNC", "JVNE", "JVNZ",
          "JVNO", "JVNP", "JVPO", "JVNS", "JVO", "JVP", "JVPE", "JVS", 
          "JMPJVOD" ]
\end{verbatim}

In order to count occurrences of the branch instructions listed
in the \texttt{jump32s} table, the semantics of the branch instructions is 
extended to increment a counter.  This is expressed in the INSTR language
as follows: 

\begin{verbatim}
jump32s label
{
    increment_counter(SSL(%pc), PARAM(label));
    SSL_INST_SEMANTICS
}
\end{verbatim}

where the function \texttt{increment\_counter} is defined in the 
\texttt{IMPLEMENTATION\_ROUTINES} section of the specification file, 
and \texttt{SSL\_INST\_SEMANTICS} refers to the semantics of the instruction 
as specified in the semantic description file SSL.  
For illustration purposes, we show the section 
\texttt{IMPLEMENTATION\_ROUTINES}, where the function 
\texttt{increment\_counter} is implemented. 
  
\begin{smallverbatim}
IMPLEMENTATION_ROUTINES

#include <map>
#include <iostream>

// map edge to execution counts
map< pair<unsigned, unsigned>, int > edge_cnt;

// increments the branch count for edge (addr1, addr2)
void increment_counter(int addr1, int addr2) {
    // construct the edge.
    pair<unsigned, unsigned> edge = 
            pair<unsigned, unsigned>(addr1, addr2);
    map< pair<unsigned, unsigned>, int >::iterator i;
    if ((i = edge_cnt.find(edge)) == edge_cnt.end()) 
    {
        // not found in map, add it and set count to 1
        backedge_cnt[edge] = 1;
    } else 
    {
        (*i).second ++;      // increment counter by 1
    }
}
\end{smallverbatim}


To build the instrumented SPARC emulator, the SLED, SSL and the 
instrumentation file (INSTR) are included as part of the build to generate 
the matching \texttt{.m} file:

\begin{verbatim}
tools/genemu -i sparc.backbranches.inst machine/sparc/sparc-core.spec 
    machine/sparc/sparc.ssl sparcemu.m
\end{verbatim}

The \texttt{-i} option includes the instrumentation file 
\texttt{sparc.backbranches.inst} into the build of the SPARC emulator.
The contents of \texttt{sparc.backbranches.cnt500.inst} is found in the 
Appendix to this Chapter.


\subsection{Syntax of the INSTR Language}

The instrumentation file consists of two main sections:
\begin{enumerate}
\item Definition, 
\item Fetch-execute cycle, and 
\item Support code.
\end{enumerate}

The definition section specifies which instructions are to be instrumented and 
their corresponding instrumentation code.  
The fetch-execute cycle section specifies what, if any, commands need to 
be executed at each iteration of the loop. 
The support code section contains additional code that the program may call 
as part of instrumentation.  This code is expressed in the C language.  
The EBNF for the language is: 

\begin{verbatim}

specification:      parts+

parts:              definition | support_code        

definition:	        DEFINITION instrm+

instrm:	            table | semantics

table:              STRING [ SLED_names ]

semantics:          (STRING parameter_list instrument_code)+ 
                    (FETCHEXECUTE instrument_code)*

parameter_list:	    STRING (, STRING)*

instrument_code:    { (action)* SSL_INSTR_SEM }

support_code:       IMPLEMENTATION_ROUTINES c_code

\end{verbatim}

where \texttt{action} contains any valid C/C++ code written by the user.  
This piece of code is attached to the instructions specified in the table 
along with the instruction's semantics.
The following special symbols may be inserted into \texttt{action}:

\begin{enumerate}
\item \texttt{SSL\_INST\_SEMANTICS}: stands for the semantics of the 
	instruction, as described in the SSL specification file, 
\item \texttt{PARAM({string})}: indicates the value of the 
	operand \texttt{{string}} of the current instruction.  E.g.
	\texttt{PARAM(label)} of the \texttt{BA} instruction is the 
	instruction's first operand.
\item \texttt{SSL(\%{register name})}: indicates one of the machine 
	registers specified in the SSL file. E.g. \texttt{SSL(\%pc)} is the 
	location holding the value of the emulated PC register.
\end{enumerate}


\section{Appendix}

\subsection{Listing of an Instrumentation File}

\begin{smallverbatim}
# File: sparc.backbranches.cnt50.inst
# Desc: This file contains the list of instructions and actions 
#       used for instrumentation.
#       The file instruments backbranches and invokes the trace 
#       builder to build hot traces used for optimisation.

DEFINITION

branch [ "BA", "BN", "BNE", "BE", "BG", "BLE", "BGE", "BL",
         "BGU", "BLEU", "BCC", "BCS", "BPOS", "BNEG", "BVC", "BVS",
         "BNEA", "BEA", "BGA", "BLEA", "BGEA",
         "BLA", "BGUA", "BLEUA", "BCCA", "BCSA", "BPOSA",
         "BNEGA", "BVCA", "BVSA" ]	
branch label 
{
    int oldpc = SSL(%pc)		
    SSL_INST_SEMANTICS
  
    // check back branches
    if (oldpc > PARAM(label)) {
        if (trace_mode) {
            end_build_trace();
		} else {		
            // branch is taken
            if (SSL(%npc) == PARAM(label)) {			
                increment_counter(oldpc, PARAM(label));
            }
        }
    }
}

FETCHEXECUTE
{
    if (trace_mode) {
        add_to_trace(SSL(%pc));		
    }
    SSL_INST_SEMANTICS
}

INSTRUMENTATION_ROUTINES
#include "emu/backbranches.cnt500.cc"
\end{smallverbatim}


\subsection{Listing of Support Code: backbranches.cnt500.cc}

\begin{smallverbatim}
#include <map>
#include <list>
#include <iostream>

// an edge is made up by a pair of addresses (branch_inst, target_inst)

// the upper value that an edge must reach to trigger the building of traces.
// this is value is incremented by 500 each time start_trace is called.
int trigger = 500;

// map edge to execution counts
map< pair<unsigned, unsigned>, int > backedge_cnt;

// informs the emulator whethere it is in trace mode.  If so, 
// it will add the current instruction at %pc to the trace.
int trace_mode = 0;

// holds the list of instructions in the current trace
list<unsigned> trace;	


// ***************************************************************************
// Function definitions
// ***************************************************************************

// prints the trace to stderr and exit		
void print_trace() {
    cerr << "Trace found: " << endl;
    list<unsigned>::iterator i;
    for (i = trace.begin(); i != trace.end(); i++) {	
        cerr << hex << (*i) << endl;
    }
    abort();
}

// signal end of trace and invoke optimizer
// at the moment it just prints the trace to the screen
void end_build_trace() {
    trace_mode = 0;
    print_trace();
}
		
// add instruction curr_inst to the trace list
// if curr_inst is the back branch that started trace building, 
// then trace building
void add_to_trace(unsigned curr_inst) {
    trace.push_back(curr_inst);
    // check to see if it is the end of trace
    if (trace.front() == curr_inst) {
        end_build_trace();
    }
}

// start building a hot trace from back branch edge
void start_trace(pair<unsigned, unsigned> edge) {
    cerr << "Starting trace finding!" << endl;	
    trace_mode = 1;
    // increase the trigger count or interate through the entire 
    // map and reset all counters or just empty the map
    trigger += 500;
    trace.clear();
    // add edge to the current trace
    trace.push_back(edge.first);
}

// increments the branch count for edge (addr1, addr2)
// if count reaches trigger, call start_trace to begin trace building.
void increment_counter(int addr1, int addr2) {
    // construct the edge.
    pair<unsigned, unsigned> edge = 
                       pair<unsigned, unsigned>(addr1, addr2);
    map< pair<unsigned, unsigned>, int >::iterator i;
    if ((i = backedge_cnt.find(edge)) == backedge_cnt.end()) {
        // not found in map, add it and set count to 1
        backedge_cnt[edge] = 1;
    } else {
        (*i).second ++;              // increment counter by 1
        if ((*i).second >= trigger) {
            start_trace(edge);
        }
    }
}
\end{smallverbatim}


\subsection{Sample Output}

Given the specification file in this Appendix, the newly generated 
\texttt{.m} file will now contain the extra instrumentation code as part 
of the emulation functions.  For example, the code for the branch always, 
\texttt{BA}, and the branch not equal, \texttt{BNE} instructions is: 

\begin{smallverbatim}
void executeBA( sint32_t reloc )
{
  int oldpc = regs.r_pc;
  
  regs.r_npc = (((1) == (0)) ? regs.r_npc : reloc);

  // check back branches
  if (oldpc > reloc) {
    if (trace_mode) {
        end_build_trace();
    } else {		
        if (regs.r_npc == reloc) {			
            increment_counter(oldpc, reloc);
        }
    }
  }
}

void executeBNE( sint32_t reloc )
{
  int oldpc = regs.r_pc;

  regs.r_npc = (((regs.r_ZF) != (0)) ? regs.r_npc : reloc);

  // check back branches
  if (oldpc > reloc) {
    if (trace_mode) {
        end_build_trace();
    } else {		
        if (regs.r_npc == reloc) {			
            increment_counter(oldpc, reloc);
        }
    }
  }
}

...

void executeOneInstruction()
{
    if (trace_mode) {
        add_to_trace(regs.r_pc);		
    }

    sint32_t tmp;
    tmp = regs.r_pc;
    regs.r_pc = regs.r_npc;
    regs.r_npc = ((regs.r_npc) + (4));
    regs.rd[0] = 0;
    execute(tmp);
}
\end{smallverbatim}


