%
% 29 Aug 01 - Cristina: created 
% 24 Oct 01 - Brian: Added testing section. Made various edits. 
%  8 Jan 01 - Cristina: updated
%

\chapter{Building Walkabout}
\label{ch-config}

{\small
\begin{flushright}
Documentation: Cristina [Aug 01, Jan 02]
\end{flushright}
}

This chapter contains notes on how to configure the \walk\ framework
for a given platform.  
The tools that can be built within the \walk\ framework include 
an emulator, a disassembler, a pathfinder and a debugger. 
Most of the examples are for the SPARC architecture as this was 
our development platform.  


\section{Compilers and Tools Needed to build \walk}
We use gcc 2.95.3, however, we do not make use of any of the new 
classes that are not available in 2.95-2, such as sstream. 
Note that we make use of namespaces sparsely in the code and these
are not supported by the gcc 2.8.1 version of the compiler, 
but they are in egcs-1.1.2 (gcc 2.91.66). 

For debugging, gdb 5.0 works well with gcc 2.95.3.


\subsection{Special tools needed to build \walk}

\walk\ has many source files that are generated from other source files,
or from specifications. It is possible to make \walk\ without installing
these tools, but if you want to make significant changes to \walk, you will
need those tools.

To make \walk\ without the special tools, use the \texttt{--enable-remote}
configuration script (see above).

The special tools are as follows.

\begin{itemize}
\item The New Jersey Machine Code Toolkit, ML version. This tool reads machine
specifications, and in association with a matcher (\texttt{.m}) file, generates
binary decoders. For details and downloading, see
\texttt{http://www.eecs.harvard.edu/~nr/toolkit/ml.html}.
\item Bison++ and Flex++, C++ versions. Note that the GNU tool bison++ is
{\it not} suitable; \walk\ needs the special versions from France, which are
C++ aware. If you get lots of errors from running bison++, you have probably
got the wrong version! Download these tools from
\texttt{ ftp://ftp.th-darmstadt.de/pub/programming/languages/C++/tools/flex++bison++/LATEST/}
You might also try downloading these tools from one of the various mirror sites such as
\texttt{http://sunsite.bilkent.edu.tr/pub/languages/c++/tools/flex++bison++/LATEST/}.
To test if you have the correct version, you should get results similar to:
\begin{verbatim}
%  bison++ --version
bison++ Version 1.21-7, adapted from GNU bison by coetmeur@icdc.fr
\end{verbatim}
If searching the web for these tooks, include the author's name ("coetmeur")
as a keyword.
\item The Tcl shell (\texttt{tclsh}). This tool is only needed to run the
regression test script (\texttt{test/regression.test}).
\texttt{tclsh} and the \texttt{tcltest} package are part of
Tcl/Tk releases 8.0 and newer.
You may well find that these are already installed on
your Linux or other system. Otherwise, see web pages such as
\texttt{http://www.sco.com/Technology/tcl/Tcl.html}.
\end{itemize}


\section{Configuration Notes}
In order to instantiate a translator out of the \walk\ framework, 
you need to configure \walk\ to run on your host machine by instantiating 
a set of source and target machines.  Figure~\ref{fig-mach-names} 
lists the names used within \walk\ to describe machine specifications, and 
the associated instruction set version that is specified.  

\centerfigbegin
\begin{tabular}{|l|l|} \hline
Name	& Description \\ \hline
sparc	& SPARC V8 (integers and floats) \\
pent	& 80386 (integers and floats) \\ \hline
\end{tabular}
\centerfigend{fig-mach-names}{Names of Machines and Versions Supported 
	by the \walk\ Framework}

You can get help from the configure program at any point in time by 
emitting the following command: 
\begin{verbatim}
   ./configure --help
\end{verbatim}

Figure~\ref{fig-config} shows the options used by \walk\ from the 
\verb!configure! program. 

\centerfigbegin
\begin{tabular}{|l|l|} \hline
Option 	& Description \\ \hline
  --enable-remote       & don't try to regenerate generated files \\
  --enable-debug[=$<what>$] & enable debugging support, $<what>$ is one of ***\\
  --with-source=$<arch>$  & translate from $<arch>$ architecture,
                          one of sparc, pent \\
  --with-instrm=$<dir>$   & add instrumentation to emulator using files in $<dir>$ \\  \hline
\end{tabular}
\centerfigend{fig-config}{Configure Options}


\section{Configuring Tools from the \walk\ Framework}
At present (Aug 2001), you can generate disassemblers, interpreters 
and a PathFinder program using the \walk\ framework. 
In order to generate these tools, the framework has to be configured
using different options.  These notes describe how to \texttt{configure} 
\walk\ for different purposes.  More information about how \texttt{configure} 
works is available in Section~\ref{sec-config-process}.  
Note that in order to build a different tool you always need to 
reconfigure your system. 


\subsection{Generating Interpreters}

In order to generate interpreters, you need to first build the tool
that generates interpreters, \texttt{genemu}: 

\begin{verbatim}
   configure --with-source=sparc --enable-remote
   make dynamic/tools/genemu
\end{verbatim}
This generates the file \texttt{genemu} in the \texttt{./dynamic/tools} 
directory. 

The \texttt{genemu} tool will create an interpreter based on 
the syntax (SLED) and semantic (SSL) specifications for a machine. 
Both a C-based interpreter and a Java-based interpreter can 
be generated using \texttt{genemu}, although the Java-based interpreter
has only been tested with the SPARC specifications. 

The options available in \texttt{genemu} are: 
\begin{verbatim}
Usage: genemu [options] <sled-filename(s)> <ssl-filename>
Recognized options:
   -c  output C code [default]
   -d  disassembler only (do not generate emulator core)
   -j  output Java code
   -i  inst-filename: use inst-filename to instrument code.
   -t  test only, no code output
   -m  additionally generate a Makefile to go with the core
   -o <file>  write output to the given file
\end{verbatim}

Note that one or more SLED files can be given as input, SLED files
have the extension \texttt{.spec} and SSL files have the extension 
\texttt{.ssl}.  Also, some options have not been maintained, it is 
best to use the configure options (see next sections) or see the 
configuration files for examples of usage.  


\subsubsection{C-based Interpreters} 
To generate a C-based interpreter for a particular machine, use the 
\texttt{make} rule for that machine
(normally the name of the machine followed by ``emu''). 
For the SPARC, you would run: 
\begin{verbatim}
   make sparcemu
\end{verbatim}
This will generate \texttt{sparcemu} in the {\texttt{./dynamic/emu} 
directory, if you had configured for the SPARC machine.

To run: 
\begin{verbatim}
   cd dynamic/emu
   sparcemu ../../test/sparc/hello 
   sparcemu /bin/banner Hi
\end{verbatim}


\subsubsection{Java-based Interpreters}
To generate a Java-based interpreter for the SPARC machine, use 
the following make rule: 
\begin{verbatim}
   make dynamic/emuj
\end{verbatim}

You can then run the generated interpreter using a Java VM: 
\begin{verbatim}
   cd dynamic/emu
   java -cp . sparcmain ../../test/sparc/hello
\end{verbatim}


\subsection{Generating PathFinder}

To build a virtual machine that finds hot paths and generates 
native code for those paths while interpreting other cold 
paths, you need to configure \walk\ to generate the \texttt{pathfinder} 
virtual machine (VM)
by configuring for a particular instrumentation method to determine
hot paths within the interpreter.  For example, if you want to use   
Dynamo's next executing tail (NET) method, run the following
configure command: 

\begin{verbatim}
   configure --with-source=sparc --enable-remote \ 
             --with-instrm=dynamic/pathfinder/sparc.NET.direct/pathfinder 
\end{verbatim}

Then make the tool by building a target
with the name formed by adding a single-character prefix
to the word ``pathfinder''.
This prefix character is the first letter of the name of the machine 
on which the genberated \texttt{pathfinder} will run.  
For example, for SPARC you would run: 

\begin{verbatim}
   make spathfinder
\end{verbatim}

This generates the instrumented VM \texttt{spathfinder} in the
\texttt{./dynamic/emu} directory.
This instrumented VM uses code profiling as well as code generation
to execute code for SPARC V8.
Note, however, that the source code for the code generator
relies on some SPARC V9 instructions. 

To run: 
\begin{verbatim}
   cd dynamic/emu
   spathfinder ../../test/sparc/hello
   spathfinder ../../test/sparc/fibo-O0
   spathfinder /bin/banner Hello
\end{verbatim}

A word of caution when building interpreters and VMs.  Some of the
files used by these tools are the same and some get patched, therefore, 
it is wise to remove object files before building a new tool.  We
therefore recommend you do a \texttt{make dynclean} before you build 
your tool.  If you are getting strange errors, most likely you need 
to \texttt{make dynclean}. 
Since you need to run configure, you can do this at the same time:

\begin{verbatim}
    make dynclean
    configure --with-source=sparc ...  [whatever other options]
\end{verbatim}

You can generate other pathfinder tools for the SPARC architecture 
by using some of the other instrumentation files.  The above example 
made use of the 
\texttt{dynamic/pathfinder/sparc.NET.direct/pathfinder/profile.inst} 
instrumentation file.  
Other instrumentation files can be used, the ones in the \walk\ 
distribution are in the following locations: 

\begin{verbatim}
   dynamic/pathfinder/sparc.NET.direct/pathfinder-call/profile.inst
   dynamic/pathfinder/sparc.NET.direct/pathfinder-recursive/profile.inst
   dynamic/pathfinder/sparc.NET.direct/pathfinder.v9/profile.inst
   dynamic/pathfinder/sparc.NET.relocate/profile.inst
\end{verbatim} 


\subsection{Generating the \walk\ Debugger} 

The \walk\ debugger is a GUI debugger written in the Java language. 
The debugger was only ever tested with the SPARC architecture, 
other extensions would be needed to support the display of state 
information for other architectures.   

To configure the debugger, emit the following commands: 

\begin{verbatim}
   make dynclean
   configure --with-source=sparc --enable-dynamic --enable-remote
   make dynamic/emuDebug
\end{verbatim}
The make builds the disassembler and the interpreter for the configured 
machine, and then generates \texttt{emuDebug.class} in the 
\texttt{./dynamic/emuDebug/bin} directory.

To run the debugger, execute the bash script \texttt{emuDebug} in the 
\texttt{./dynamic/emuDebug} directory (this is a shell script file; 
make sure the first line refers to the location of your \texttt{bash} tool):  

\begin{verbatim}
   cd dynamic/emuDebug
   emuDebug ../../test/sparc/hello 
\end{verbatim}


\subsection{Building without the \texttt{--with-remote} Option}

When configuring the system \emph{without} use of the remote option 
(i.e. without \texttt{--with-remote}), the system will recreate .m and .cc 
files from the machine specifications and .c files from .y files.  

It is recommended that you do a \texttt{make realdynclean} before 
configuring without the remote option, in order to remove all generated
files that have already been stored in the distribution. 


\section{How the Configuration Process Works}
\label{sec-config-process}
A complete description of the autoconfigure process is beyond the scope of
this document; the interested reader can get more information from
publicly available
documentation such as \texttt{http://www.gnu.org/manual/autoconf/index.html}.

In brief, the developer writes a file called \texttt{configure.in}. The program
\texttt{autoconf} processes this file, and produces a script file called
\texttt{configure} that users run to configure their system. We have
already done that, so unless you need to change the configuration, you
only need to run \texttt{./configure}. If you do make a change to
\verb!configure.in!, then you should run
\begin{verbatim}
   autoconf; autoheader
\end{verbatim}

When \texttt{./configure} is run, various files are read, including a file
specific to the source machine. For example, if you configure with
\texttt{--with-source=sparc}, the file \texttt{machine/sparc/sparc.rules}
is read for SPARC-specific information. It also reads the file
\texttt{Makefile.in}. Using this information and the command line options,
\texttt{./configure} creates the file \texttt{Makefile}.
As a result, the \texttt{Makefile} isn't even
booked in. That's the main reason you need to run \texttt{./configure} as
the very first thing, before even \texttt{make}. It also means that you
should not make changes (at least, changes that are meant to be permanent)
to \verb!Makefile!; they should be made to \texttt{Makefile.in}.

Another important file created by \verb!./configure! is \verb!include/config.h!.
This file is included by \verb!include/global.h!, which in turn is included
by almost every source file. Therefore, \verb!configure! goes to some trouble
not to touch \verb!include/config.h! if there is no change to it (and it says
so at the end of the \verb!configure! run). A significant change to the
configuration (e.g., choosing a new source or target machine) will
change \verb!include/config.h!, and therefore almost everything will
have to be recompiled.

A note about the version of \texttt{autoconf}; we have found that 
version 2.9 does not work properly but version 2.13 works fine with 
our \texttt{configure} files. 


\subsection{Dependencies and \texttt{make depend}}
Building \walk requires a file called \verb!.depend!
that contains file dependencies for the system.
The first time you \verb!make! \walk,
this file won't exist
and it will be created automatically for you.
The \verb!.depend! file contains entries similar to this:

\begin{verbatim}
coverage.o: ./coverage.cc include/coverage.h include/global.h \
 include/config.h
\end{verbatim}

which says that the \verb!coverage.o! file depends on the files
\verb!./coverage.cc!, \verb!include/coverage.h!, and so on. There can be
dozens of dependencies; the above is one of the smallest. This information
takes a minute or two to generate, and so is only generated (a) by \verb!make!
itself if 
\verb!.depend! does not exist, and (b) if the user types \verb!make depend!.

It is easy to change the dependencies, e.g. by adding a
\verb!#include! line to a source file. If you do this, and forget to run
\verb!make depend!, then you can end up with very subtle make problems that
are very hard to track down. For example, suppose you add
``\verb!#include "foo.h"!'' to the \verb!worker.cc! source file,
so that \verb!worker.cc! can use the last virtual method in class \verb!foo!.
Everything compiles and works fine. A week later, you add a virtual method to
the middle of \verb!class foo!. The \verb!.depend! file doesn't have the
dependency for \verb!worker.cc! on \verb!foo.h!, and so \verb!worker.o! isn't
remade. The code in \verb!worker.o! now calls the second last method
in \verb!class foo!, instead of the correct final method! However,
you are not thinking about \verb!worker.cc! now, since your latest changes
are elsewhere. This sort of problem can take a long time to diagnose.

One solution is to ``\verb!make clean!'' as soon as you get unexpected
results. However, you can save a lot of time if instead you just
\verb!make depend; make! instead. In fact, it's a good idea to run
\verb!make depend! regularly, or after any significant change to the source
files.

\subsection{Warnings from \texttt{make}}

During the making of \walk, it is normal to see quite a lot of output. We try to
ensure that ordinary warnings from gcc are prevented, but some warnings are much
harder to suppress, and some warnings are quite normal. For example:

\verb!typeAnalysis/typeAnalysis.y contains 2 shift/reduce conflicts.!

These are normal, and the bison++ parser automatically resolves these conflicts
in a sensible way.


\subsection{Where the Makefile Rules Are} 
The Makefile is composed of make rules
that come from a variety of different sources:

\begin{itemize}
\item The core rules are in the top level \texttt{Makefile.in} file, 
\item Machine-specific rules are in the respective machine directory 
	with the extension \texttt{.rules};
      e.g., \texttt{machine/sparc/sparc.rules},
\item Instrumentation rules are in a subdirectory \texttt{pathfinder/make.rules}
	under the respective instrumentation directory; e.g.,
	\texttt{pathfinder/sparc.NET.direct/pathfinder/make.rules}. 
\end{itemize}


\section{The \walk\ Regression Test Suite}
The \walk\ framework includes a set of regression tests for
generated interpreters.
Tests have been written for the SPARC emulator only.

The script that is used for testing an interpreter is called 
\texttt{dynamic/test/interp-regression.tcl}.
This is a Tcl script that allows new tests to be added easily.
Existing tests can also be modified easily.
Each test includes the test's name and expected result.
The script runs each test and compares its output against that expected.
If these are not the same, it reports the failure and gives the actual result.
The default is to run all tests,
but you can specify which tests to run,
or which to skip, by giving a regular expression pattern
that is matched against test names.
At the end of all the tests, there is a report
on the number of tests run, and how many passed, failed, or were skipped.

The tests each run one of the SPEC95 benchmark programs.
Up to four regression tests can be run for each SPEC95 program.
Both optimized (-O4) and unoptimized (-O) versions of each program are run.
It is also possible to run each version of the program
using both the SPEC95 ``test'' and ``reference'' data sets.
By default, only the ``test'' data set is used
since it requires less time.
The other tests that use the ``reference'' data set
are marked as having the ``refInput'' contraint;
they are only run if you specify \texttt{-constraints refInput}
on the command line that runs the regression tests (see below).

The expected output of each test
is the total number of instructions executed.
This count is sensitive to the specific versions of the shared libraries that
are used.
As a result, you can expect the tests to fail if they are run on
a machine other than that used to get the original expected instruction count.
We used a Sun 420R with 4 CPUs and 4GB of memory, running Solaris 8. 


\subsection{Running the Regression Tests} 
Tests can be run sequentially (one after another) or in parallel.
This section discusses how to run the tests sequentially.
Section~\ref{sec-mt-tests} describes how to
run the tests concurrently.

To run all SPARC interpreter tests using the ``test'' data sets (the default):
\begin{verbatim}
    $ cd <workspace>
    $ cd dynamic/test
    $ tclsh interp-regression.test sparc
\end{verbatim}

To run only the SPARC interpreter tests that run the SPEC95 ``go'' program,
you can specify a pattern on the command line.
Only tests (``test'' data sets only)
with names that match the pattern are run:
\begin{verbatim}
    cd dynamic/test
    tclsh interp-regression.test sparc -match 'go*'
\end{verbatim}

To skip all SPARC interpreter tests whose names match a pattern
(again ``test'' data sets only), run:
\begin{verbatim}
    cd dynamic/test
    tclsh interp-regression.test sparc -skip 'go* ijpeg*'
\end{verbatim}

To run all SPARC interpreter tests
including those that use the ``reference'' data sets:
\begin{verbatim}
    cd dynamic/test
    tclsh interp-regression.test sparc -constraints refInput
\end{verbatim}


\subsection{Running the Tests in Parallel} 
\label{sec-mt-tests}

To run the tests in parallel,
the two scripts \texttt{dynamic/test/mt-tests.tcl}
and \texttt{dynamic/test/summarize-mt-tests.tcl}
are used.
\texttt{mt-tests.tcl} takes the same command line arguments
as \texttt{interp-regression.tcl}
and forks processes to do the various tests in parallel.
It creates a directory under
\texttt{dynamic/test} with a name like \texttt{walktest-Oct-24-2001-19:15:06}
that contains an ``*.out'' file holding the output
from each of the forked tests.
One such file, for example, might be \texttt{go.v8.base.test.out}.

The second script \texttt{summarize-mt-tests.tcl}
creates a file that summarizes the results of all the tests.
It takes one command line argument,
the name of the directory created by \texttt{mt-tests.tcl},
and creates a file \texttt{summary.txt} in that
directory with the concatenated results of the various tests.

For example, to run all SPARC interpreter tests except for ``go''
using the default ``test'' data sets, do the following.

\begin{verbatim}
    $ cd <workspace>/dynamic/test
    $ tclsh mt-tests.tcl sparc -skip 'go*'
    Parallel Walkabout tests
    Writing results into directory walktest-Oct-24-2001-16:10:44
    Forked process 7952 to execute test compress.v8.peak.test
    Forked process 7953 to execute test compress.v8.base.test
    Forked process 7954 to execute test perl.v8.peak.test
    Forked process 7955 to execute test perl.v8.base.test
    Forked process 7956 to execute test go.v8.peak.test
    Forked process 7957 to execute test go.v8.base.test
    Forked process 7958 to execute test ijpeg.v8.peak.test
    Forked process 7964 to execute test ijpeg.v8.base.test
    $ tclsh summarize-mt-tests.tcl walktest-Oct-24-2001-16:10:44
    Summarizing parallel Walkabout test results into walktest-Oct-24-2001-16:10:44/summary.txt
    appending file compress.v8.base.test.out
    ...
    $ vi walktest-Oct-24-2001-16:10:44/summary.txt
\end{verbatim}

